
%%%%%%%%%%%%%%%%%%%%%%%%%
%    Summary 
%%%%%%%%%%%%%%%%%%%%%%%%%
\documentclass[12pt]{article}

\oddsidemargin 0.10in \evensidemargin -0.65in
\textwidth 6.2in         % Width of text line.
\topmargin 0.60in \headheight 0.0in \headsep 0.0in
\textheight 8.5in        % Height of text (including footnotes and figures,
\topskip 0.0in

%  \usepackage{showkeys}
\usepackage{color}
\usepackage{amsmath, amssymb, latexsym, natbib}
\usepackage{psfrag,epsfig,amsfonts,amsmath,latexsym,amsthm,amssymb,amscd,url }



  % \includeonly{description2012}
  %  \includeonly{ref1009}



\newcommand{\F}{{\mathcal{F}}}
\newcommand{\B}{{\mathcal{B}}}

\newcommand{\eps}{\varepsilon}



\renewcommand{\k}{\kappa}
\newcommand{\p}{\partial}
\newcommand{\D}{\Delta}
\newcommand{\om}{\omega}
\newcommand{\Om}{\Omega}
\renewcommand{\phi}{\varphi}
\newcommand{\e}{\epsilon}
\renewcommand{\a}{\alpha}
\renewcommand{\b}{\beta}
\newcommand{\N}{{\mathbb N}}
\newcommand{\R}{{\mathbb R}}
\newcommand{\T}{{\mathbb T}}

\newcommand{\Le}{L_t^{\alpha}}

\newcommand{\EX}{\mathbb{E}}
\newcommand{\PX}{\mathbb{P}}


\newcommand{\grad}{\nabla}
\newcommand{\n}{\nabla}
\newcommand{\curl}{\nabla \times}
\newcommand{\dive}{\nabla \cdot}

\newcommand{\ddt}{\frac{d}{dt}}
\newcommand{\la}{{\lambda}}

\newcommand{\bu}{\mathbf{u}}

\newcommand{\obu}{\bar{\mathbf{u}}}
\newcommand{\bsigma}{\mathbf{\sigma}}
\newcommand{\btau}{\mathbf{\tau}}


\newcommand{\nd}{{\nabla \cdot}}

\newcommand{\cF}{{\cal F}}
\newcommand{\cG}{{\cal G}}
\newcommand{\cD}{{\cal D}}
\newcommand{\cO}{{\cal O}}

%%%%%%%%%%%%%%

\newtheorem{theorem}{Theorem}
\newtheorem{lemma}{Lemma}
\newtheorem{definition}{Definition}
 \newtheorem{coro}[lemma]{Corollary}
 \newtheorem{example}[lemma]{Example}
 \newtheorem{remark}[lemma]{Remark}


%%%%%%%%%%%%%%% Xu Sun %%%%%%%%%%%%%



\newcommand{\uk}[1]{\ensuremath{u^{(#1)}(t,\omega)}}
\newcommand{\hse}{\ensuremath{h^s(\xi,\omega)}}
\newcommand{\hsk}[1]{\ensuremath{h^{(#1)}(\xi,\omega)}}

\newcommand{\sz}{\ensuremath{ {\int_s^0 z(\theta_r (\omega))\,dr}}}
\newcommand{\sZ}{\ensuremath{ {\int_s^0 Z(\theta_r (\omega))\,dr}}}
\newcommand{\zz}[1]{\ensuremath{{z(\theta_{#1} (\omega))}}}
\newcommand{\ZZ}[1]{\ensuremath{{Z(\theta_{#1} (\omega))}}}
\newcommand{\fu}[1]{\ensuremath{{F_u^{u_0 (#1)}}}}
\newcommand{\fus}[2]{\ensuremath{{\int^0_{#2} F_u^{u_0 (#1)}\,d{#1}}}}
\newcommand{\fuss}[2]{\ensuremath{{\int^{#2}_0 F_u^{u_0 (#1)}\,d{#1}}}}

\newcommand{\fuu}[1]{\ensuremath{{F_{uu}^{u_0(#1)}}}}
\newcommand{\rb}{\right)}
\newcommand{\lb}{\left(}
\newcommand{\rB}{\right]}
\newcommand{\lB}{\left[}


\newcommand{\nb}{\mathbf{n}}
\newcommand{\ub}{\mathbf{u}}
\newcommand{\xb}{\mathbf{x}}
\newcommand{\xnb}{\mathbf{x}_0}
\newcommand{\GaB}{\mathbf{\Gamma}}

\newcommand{\bo}{\mathcal {O}}
\newcommand{\so}{\mathcal {o}}

\newcommand{\BE}{\begin{equation}}
\newcommand{\EE}{\end{equation}}
\newcommand{\BEN}{\begin{equation*}}
\newcommand{\EEN}{\end{equation*}}
\newcommand{\BAL}{\begin{align}}
\newcommand{\EAL}{\end{align}}
\newcommand{\BAN}{\begin{align*}}
\newcommand{\EAN}




\begin{document}
Mathematical modeling and computation are of critical importance in addressing contemporary scientific and engineering challenges. By providing to junior scholars the opportunity to work in interdisciplinary teams of mathematicians, engineers, chemists and neuro-scientists studying complex, large scale systems, the Southern Methodist University Research Training Group (SMU-RTG) will develop the trainees skills in modeling and computation. In addition, the SMU-RTG will increase the number of US citizens and permanent residents undertaking advanced studies in computational and applied mathematics and pursuing careers both in and out of academia. The SMU-RTG will pursue this goal through the formation of three vertically-integrated research training groups, each of which will partner SMU mathematicians with scientists in other disciplines. Moreover, by partnering on a summer undergraduate research program with the University of Texas Rio Grande Valley (UTRGV - a Hispanic-serving institution), more students from underrepresented groups will be encourage to undertake graduate studies in STEM fields.

The research groups involved in the SMU-RTG will address challenging problems in emerging scientific fields: data-driven models in neuroscience, the nonlinear dynamics of the electric power grid and other large-scale networks, and the fabrication and modeling of nanoscale structures. The research activities will create opportunities for the trainees through strategic partnerships with both academic and non-academic institutions: UT Southwestern Medical Center, DOE laboratories, and SMU Engineering. A centerpiece of the year-round activities will be a 10-week summer undergraduate research program with participants from both SMU and UTRGV. Each group will begin its program with a 2-week summer school, which will introduce undergraduates, graduate students and postdocs to fundamental concepts and techniques used by one of the research groups. This activity will provide exciting and substantive research experiences to students of all backgrounds. The Department of Mathematics at SMU is well-positioned to execute both the research and educational goals of this project, building on its core strengths in modeling and scientific computing and its ongoing success in placing graduates in a range of careers, from industry to national laboratories. The research themes reflect both the department's existing strengths and pressing societal needs, carrying out research that will contribute to improved understanding of the human brain in both health and disease, the reliability of crucial infrastructure, and novel devices with applications in medicine and beyond.


\end{document}
