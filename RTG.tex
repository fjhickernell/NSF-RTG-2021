
%%%%%%%%%%%%%%%%%%%%%%%%%
%   NSF RTG  2021
%  
%
%   Due:   June 1, 2021
%%%%%%%%%%%%%%%%%%%%%%%%%
\documentclass[11pt]{NSFamsart}

%%%%%%%%%%%%%%% Margins and Page Numbers %%%%%%%%%%%%%%%%%%%%%%%%%%%%%% 
\voffset 0.23in  %Made to satisfy Research.gov compliance
\hoffset 0.01in
\textheight 8.91in
\textwidth 6.46in
\setlength{\oddsidemargin}{0in}
\setlength{\evensidemargin}{0in}
%\thispagestyle{empty} \pagestyle{empty} %to eliminate page numbers for upload
\thispagestyle{plain} \pagestyle{plain} %to add back page numbers
%%%%%% Squeeze the space %%%%%%%%%%%%%%%%%%%%%%%%%%%%%% 
%\renewcommand{\baselinestretch}{0.97}
%%%%%%Squeeze the space %%%%%%%%%%%%%%%%%%%%%%%%%%%%%% 

\headsep-0.6in

%%%%%%%%%%%%%%% Margins and Page Numbers %%%%%%%%%%%%%%%%%%%%%%%%%%%%%% 

%\oddsidemargin 0.10in \evensidemargin -0.65in
%\textwidth 6.2in         % Width of text line.
%\topmargin 0.60in \headheight 0.0in \headsep 0.0in
%\textheight 8.5in        % Height of text (including footnotes and figures,
%\topskip 0.0in

%  \usepackage{showkeys}
\usepackage{color}
\usepackage{amsmath, amssymb, latexsym, natbib, xspace}
\usepackage{psfrag,epsfig,amsfonts,amsmath,latexsym,amsthm,amssymb,amscd,url }
\usepackage{amsmath}
\usepackage{bm}
\usepackage{enumitem}
 
\setlist[description]{font=\normalfont\bfseries, labelindent = 0cm, leftmargin=0ex}
\setlist[itemize]{leftmargin=5ex}


\newcommand{\F}{{\mathcal{F}}}
\newcommand{\B}{{\mathcal{B}}}
\providecommand{\HickernellFJ}{Hickernell}

\newcommand{\eps}{\varepsilon} 

\renewcommand{\k}{\kappa}
\newcommand{\p}{\partial}
\newcommand{\D}{\Delta}
\newcommand{\om}{\omega}
\newcommand{\Om}{\Omega}
\renewcommand{\phi}{\varphi}
\newcommand{\e}{\epsilon}
\renewcommand{\a}{\alpha}
\renewcommand{\b}{\beta}
\newcommand{\N}{{\mathbb N}}
\newcommand{\R}{{\mathbb R}}
\newcommand{\T}{{\mathbb T}}

\newcommand{\Le}{L_t^{\alpha}}

\newcommand{\EX}{\mathbb{E}}
\newcommand{\PX}{\mathbb{P}}


\newcommand{\grad}{\nabla}
\newcommand{\n}{\nabla}
\newcommand{\curl}{\nabla \times}
\newcommand{\dive}{\nabla \cdot}

\newcommand{\ddt}{\frac{d}{dt}}
\newcommand{\la}{{\lambda}}

\newcommand{\bu}{\mathbf{u}}

\newcommand{\obu}{\bar{\mathbf{u}}}
\newcommand{\bsigma}{\mathbf{\sigma}}
\newcommand{\btau}{\mathbf{\tau}}


\newcommand{\nd}{{\nabla \cdot}}
\newcommand{\dd}{\mathrm{d}}
\newcommand{\x}{{\bm x}}

\newcommand{\cF}{{\cal F}}
\newcommand{\cG}{{\cal G}}
\newcommand{\cD}{{\cal D}}
\newcommand{\cO}{{\cal O}}

\newcommand{\numUG}{65\xspace}
\newcommand{\numPhD}{21\xspace}
\newcommand{\numPostDoc}{3\xspace}

%%%%%%%%%%%%%%

\newtheorem{theorem}{Theorem}
\newtheorem{lemma}{Lemma}
\newtheorem{definition}{Definition}
 \newtheorem{coro}[lemma]{Corollary}
 \newtheorem{example}[lemma]{Example}
 \newtheorem{remark}[lemma]{Remark}


%%%%%%%%%%%%%%% %%%%%%%%%%%%%



\newcommand{\uk}[1]{\ensuremath{u^{(#1)}(t,\omega)}}
\newcommand{\hse}{\ensuremath{h^s(\xi,\omega)}}
\newcommand{\hsk}[1]{\ensuremath{h^{(#1)}(\xi,\omega)}}

\newcommand{\sz}{\ensuremath{ {\int_s^0 z(\theta_r (\omega))\,dr}}}
\newcommand{\sZ}{\ensuremath{ {\int_s^0 Z(\theta_r (\omega))\,dr}}}
\newcommand{\zz}[1]{\ensuremath{{z(\theta_{#1} (\omega))}}}
\newcommand{\ZZ}[1]{\ensuremath{{Z(\theta_{#1} (\omega))}}}
\newcommand{\fu}[1]{\ensuremath{{F_u^{u_0 (#1)}}}}
\newcommand{\fus}[2]{\ensuremath{{\int^0_{#2} F_u^{u_0 (#1)}\,d{#1}}}}
\newcommand{\fuss}[2]{\ensuremath{{\int^{#2}_0 F_u^{u_0 (#1)}\,d{#1}}}}

\newcommand{\fuu}[1]{\ensuremath{{F_{uu}^{u_0(#1)}}}}
\newcommand{\rb}{\right)}
\newcommand{\lb}{\left(}
\newcommand{\rB}{\right]}
\newcommand{\lB}{\left[}


\newcommand{\nb}{\mathbf{n}}
\newcommand{\ub}{\mathbf{u}}
\newcommand{\xb}{\mathbf{x}}
\newcommand{\xnb}{\mathbf{x}_0}
\newcommand{\GaB}{\mathbf{\Gamma}}

\newcommand{\bo}{\mathcal {O}}
\newcommand{\so}{\mathcal {o}}

\newcommand{\BE}{\begin{equation}}
\newcommand{\EE}{\end{equation}}
\newcommand{\BEN}{\begin{equation*}}
\newcommand{\EEN}{\end{equation*}}
\newcommand{\BAL}{\begin{align}}
\newcommand{\EAL}{\end{align}}
\newcommand{\BAN}{\begin{align*}}
\newcommand{\EAN}

\newcommand{\s}{{\sigma}}
\def\Tr{\mbox{Tr}}
\newcommand{\Rn}{\mathbb{R}^n}

\DeclareMathOperator*{\argmax}{arg\,max}
\DeclareMathOperator*{\argmin}{arg\,min}
\newcommand{\FredNote}[1]{{\color{blue} Fred: #1}}
\newcommand{\bfparagraph}[1]{\paragraph{\textbf{\textup{#1}}}}

\iffalse
The long-range goal of the Research Training Groups in the Mathematical Sciences (RTG) program is to strengthen the nation's scientific competitiveness by increasing the number of well-prepared U.S. citizens, nationals, and permanent residents who pursue careers in the mathematical sciences, be they in academia, government, or industry. A significant part of this goal is to directly increase the proportion and the absolute number of U.S. students at the RTG sites who pursue graduate studies and complete advanced degrees in the mathematical sciences. It is anticipated that RTG projects also will serve as national models for research training in the mathematical sciences. Activities with potential impact beyond the directly-supported students and beyond the institutions receiving RTG funds will be key strengths in proposals. Collaborative proposals involving different types of programs (for example, institutions in which the relevant department does not award Ph.D.s, minority-serving institutions, etc.) and having the potential to develop innovative approaches to research training in the mathematical sciences are welcome. For such collaborative efforts, the lead institution must grant a doctoral degree in mathematical sciences.

The RTG program supports efforts to improve research training by involving undergraduate students, graduate students, postdoctoral associates, and faculty members in structured research groups anchored in a coherent research program. The activities need not be focused on a particular research problem; rather, it is expected that group participants will be united by common topical interests. The groups may include researchers and students from different departments and institutions, but the research-based training and education activities must be based in the mathematical sciences. RTG projects are expected to vary in size, scope, and proposed activities, as well as in their plans for organization, participation, and operation. However, research groups supported by RTG will include vertically-integrated activities that span the entire spectrum of educational levels from undergraduates through postdoctoral associates.
Addressing all stages (from undergraduate through postdoctoral) of trainee involvement is essential in RTG proposals. Proposals that focus on only one stage are not appropriate for submission to the RTG activity. While emphasis on graduate training in RTG projects is appropriate and natural, a substantial plan for involving undergraduates is necessary. When used in reference to undergraduates, the word "research" should be given its broadest interpretation.

Successful proposals will include collaborating faculty with a history of research accomplishments. This group should have a history of working with students and/or postdoctoral associates, and they should present a strong plan for recruiting students who are U.S. citizens, nationals, or permanent residents into their program. The RTG program is not meant to establish new research groups, but to enhance the training activities of existing groups with strong research records.

Graduate Traineeships. Graduate trainees form a pivotal component of the integration of activities in RTG grants. Their participation should result in:
1. involvement with research activities that include undergraduates, other graduate students, postdoctoral associates, and/or faculty members; 
2. graduate education that is both broad and deep; and
3. significant teaching or other professional experience such as industry/laboratory internship.
Mentoring, that is, guidance in professional development, is a critical strategy for preparing graduate trainees to become successful researchers, communicators, and mentors. Graduate trainees are expected to have substantial mentored professional experiences to prepare them for successful careers in the mathematical sciences and in other professions in which expertise in the mathematical sciences plays an important role. Examples of this professional experience could include:
• a minimum of two terms of supervised teaching, preferably with one term of more independent teaching in which the student has substantial responsibility for a class, or
• a minimum of two terms of a supervised industry/laboratory internship.
Some element of their activities should help students develop proficiency in the presentation of mathematical sciences research in both written and oral formats and in the ability to place their research in context.

RTG awards are intended to allow graduate students significant time for research, course work, and related activities. A graduate trainee can receive up to 33 months of non-teaching support from an RTG activity. RTG stipends cannot be used to pay students to fulfill teaching duties or for internships. Departments must demonstrate how the traineeships will improve the quality of the education their graduate students receive. The traineeships are not intended to replace existing institutional funding of research fellowships or scholarships.

Undergraduate Experience. In this program solicitation, the term "research experiences" for undergraduates includes all activities that involve undergraduates in discovery and generate appreciation of and excitement about research in the mathematical sciences. An undergraduate research experience does not have to result in the publication of a paper. Examples of research experiences include faculty-directed projects, either during the academic year or the summer, or participation in research teams with graduate students and/or postdoctoral associates. Such experiences are intended to involve students in the creative aspects of mathematical sciences in a non-classroom setting. They are also expected to enhance the development of students' communication skills, with particular emphasis on the presentation of mathematical concepts in both written and oral formats. In all cases, it is expected that the participating undergraduates receive mentoring to stimulate their further interest in the mathematical sciences.

Postdoctoral Associates. Effective RTG activities better prepare postdoctoral associates for their future careers. It is expected that at the end of the postdoctoral experience, each associate will have a well-defined independent research program, well-developed communication skills, a broad perspective of his or her field, and the ability to mentor.

The postdoctoral program can provide opportunities not traditionally found in mathematical sciences education and training, including interdisciplinary research experiences in connection with other departments and programs; participation in international research programs; internships in business, industry, or government laboratories; or participation in research institute programs suitably aligned with the associate's research interests. Postdoctoral associates are expected to teach, on average, one course per term while in residence at the sponsoring university. Over the duration of the postdoctoral appointment, this teaching should encompass a diverse set of instructional experiences at different levels of the curriculum. Likewise, it is expected that each RTG postdoctoral associate will submit a research proposal to a funding agency at some time during the course of the postdoctoral appointment. Mentoring to help ensure all postdoctoral associates become successful researchers, communicators, and mentors is a critical element of an RTG postdoctoral program, as is interaction of postdoctoral associates with undergraduate and/or graduate students.

The typical RTG postdoctoral appointment is for three years. A person is eligible for only one RTG postdoctoral appointment. An RTG postdoctoral associate is expected to be a recent recipient of a doctoral degree, typically held not more than three years as of January 1 of the year in which the appointment begins. Any exceptions made to this restriction should be well-justified in the annual reports.

Budget. Proposals may include support requests for graduate and advanced undergraduate students, postdoctoral associates, visitors, consultant services, travel, conferences, and workshops. Other budget items that are deemed to be essential to the success of the proposed activities may be included. Faculty salary is limited to that needed for the purpose of organizing and managing the program.

Data. RTG proposals will be strengthened by supporting data about the department's programs. An extensive discussion of the requested data appears in the Supplementary Documentation section below (V.A.7).

A successful RTG proposal will:
• be based in a U.S. IHE that grants the Ph.D. in the mathematical sciences (faculty and trainees from other types of institutions may be included through a collaborative proposal or other mechanisms);
• be anchored in a coherent research program in the mathematical sciences;
• have a realistic plan showing how the proposed activity would create new or enhanced research-based training experiences in the mathematical sciences for the students and postdoctoral associates;
• be directed by a principal investigator, with at least two other faculty members, who will collaborate in management and participate fully in the RTG activities.
A successful RTG proposal must convince reviewers that the project:
• integrates research with educational activities;
• provides for developing professional and personal skills, such as communication, teamwork, teaching, mentoring, and leadership;
• includes an administrative plan and organizational structure that ensures effective management of the project resources;
• has an institutional commitment to furthering the plans and goals of the RTG project and to create a supportive environment for integrative research and education;
• has a plan for recruitment, selection, and retention of participants, including members of underrepresented groups, so as to increase the number and diversity of U.S. citizens, nationals, and permanent residents in the graduate and postdoctoral programs;
• serves as a national model by effectively disseminating best practices for attraction, retention, and high-quality preparation of students and postdoctoral associates in the mathematical sciences; and
• has a post-RTG plan. The RTG program is intended to help stimulate and implement permanent positive changes in research training within the mathematical sciences in the U.S. Thus it is critical that an RTG site adequately plan how to continue the pursuit of RTG goals when funding terminates.
\fi

%%%%%%%%%%%%%%%%%%%%%%%%%%%%%%%%%%%%%%%%%%%%%%%%%%%%%%%%%%%%%%%%%%%%%%%%%%%%%%%%%%%%%%%%%%%%%%%%%%
\begin{document}  

%\centerline{\large \textbf{RTG: Research and Training in Complex Dynamical Systems}}
\section{Introduction}

Although the basic undergraduate and graduate level mathematics curriculum provides sophisticated quantitative tools, it generally does not develop in students the \emph{mathematical maturity} to integrate these tools into a unified \emph{creative toolbox}. Developing this maturity is the overarching goal of this Research Training Grant (RTG).  

We will accomplish our goal by providing training that supports the strong, underlying linking theme of \emph{complex dynamical systems}.  These dynamical systems may be under random influences \cite{Arnold, DuanBook2015}, far from equilibrium \cite{liu2009introduction} and/or operate in quantum regimes \cite{Dittrich2016}. Our approaches for understanding them may be particle (i.e., trajectory and sample path) based or continua (i.e., probability density, ensemble, energy, action functional, and wave) based.  Our training in state-of-the-art mathematical and computational tools will introduce to a spectrum of applications that is fundamental to current scientific and technological progress, such as geophysical and climate systems, liquid crystals, multi-phase flows, ionic fluids,  transportation systems, sensor networks, and biochemical reaction systems. 

\subsection*{Environment} The PIs for this project are four Illinois Tech applied mathematics faculty,  Jinqiao Duan, Fred Hickernell, Chun Liu, and Michael Pelsmajer, plus Romit Maulik (an Argonne National Laboratory scientist with an Illinois Tech faculty appointment). This RTG builds on the research expertise among the PIs, and our existing applied mathematics curricula at Illinois Tech.  The research portfolio of our department covers a broad range of applied mathematics and statistics, and includes topics germane to this RTG, including dynamical systems, stochastic dynamical systems, multiscale and variational methods,  Monte Carlo methods, foundation of computational mathematics, and data-driven predictive modeling and simulation. Our work has applications in the physical, chemical, geophysical, biophysical, biological, and materials sciences, as well as finance.  Given that the PIs include our department chair, our former department chair, and our former associate chair, we expect to the bulk of our department to participate in this RTG in some capacity, whether as part of the research groups or implementing the curricular and co-curricular improvements.

We currently have \numUG BS applied math majors, \numPhD PhD students and \numPostDoc postdoctoral fellows or visiting assistant professors. Our graduates build their careers in  both industry  and academia. Many of our undergraduates continue for PhD studies in other top universities such as Cornell, NYU, UCLA and UIUC. Most of our postdocs or visiting assistant professors get tenure track positions after leaving Illinois Tech.


\subsection*{Partnership with Argonne, IMSI, and C2ST}
This RTG will benefit from our ongoing collaboration with Argonne National Laboratory (18 miles west of our campus). We will participate in joint research and workshops, as well as Argonne hosting worthy students for internships.  We will also begin a new collaboration with the NSF Institute for Mathematical and Statistical Innovation (IMSI), hosted at University of Chicago (5 miles south of our campus). We will encourage our RTG trainees to participate in IMSI programs, and we will co-sponsor programs of mutual interest with IMSI.  For more than a decade, the Chicago Council on Science and Technology (C2ST) has enhanced the public’s understanding and appreciation of science and technology and their impact on society. C2ST will co-sponsor outreach programs with our RTG and will train our students in communicating their scientific ideas to a broader audience.
 


%%%%%%%%%%%%%%%%%%%%%%%%%%%%%%%%%%%%%%%%%%%%%%%%%%%%%%%%
\section{Research Projects } \label{sec:researchproblems}
%%%%%%%%%%%%%%%%%%%%%%%%%%%%%%%%%%%%%%%%%%%%%%%%%%%%%%%%

We propose three research projects under the umbrella of complex dynamical systems: 1.\ dynamics under uncertainty; 2.\  energetic variational approaches for complex dynamics; and 3.\ mathematical tools for quantum dynamics. Project 1 connects with Project 2 via minimizing action functionals. In Project 3, we examine the interplay of quantum uncertainty and external fluctuations, and further consider a Bohmian trajectory formulation of quantum dynamics, thus linking with Project 1. 

The team for each project will consist of researchers with at least one member
from Argonne National Lab or departments outside applied mathematics, together with math undergraduate and graduate students, and a postdoctoral
fellow. In addition to those trainees (postdocs and students) supported directly by this RTG, other students, postdocs, and faculty members will be encouraged to join these projects.  The key faculty and scientists are listed for each project below.

\iffalse  
 While there   may be thematic driven differences on the approach to deliver
knowledge and do research, each group share a similar year-round schedule that will allow awareness
and exchange of ideas throughout the academic year, mainly at the RTG graduate seminar.   Finally each summer program is centered on
an intense 2-week period. \FredNote{What does this mean?} Participating undergraduates will have a unique opportunity to work in a serious
way on two topics, which we believe to be highly beneficial.
\fi
 

%%%%%%%%%%%%%%%%%%%%%%%%%%%%%%%%%%%%%%%%%%%%%%%%%%%%%%%%
\subsection*{Project 1. Dynamics  Under Uncertainty:} 
%%%%%%%%%%%%%%%%%%%%%%%%%%%%%%%%%%%%%%%%%%%%%%%%%%%%%%%%
Jinqiao Duan (leader), Fred Hickernell, Michael Pelsmajer, Romit Maulik, Mustafa Bilgic (Illinois Tech Computer Science) and Prasanna Balaprakash (Argonne National Lab). 

\subsubsection*{Background} The interactions of uncertainty and nonlinearity lead to intriguing phenomena, such as  transitions   between  different dynamical regimes. We develop modeling, analytical, computational and data science  methods to quantify     stochastic dynamics, with applications to chemical \cite{agaoglou_chemical_2019}, biophysical \cite{Ruoff2018BiologicalCR}, powergrid infrastructures \cite{MEDJROUBI201714}, and climate systems \cite{Alexandrov2020NonlinearCD, Franzke2017NonlinearAS, Wan2020ADF}. 
 
Complex systems are oftentimes under the influence  of randomness or uncertainty \cite{Moss1, Horst, Gar, VanKampen3}. Uncertainties may also be caused by our lack of knowledge of some physical processes that are not well represented in the mathematical models (epistemic) and due to the inherent randomness of a specific event  \cite{Palmer1, Kantz, Wilks, Williams}.
Although these random mechanisms appear to be very small or very fast, their long time impact on the system evolution may be delicate or even profound \cite{Arnold, DuanBook2015}. These delicate impacts on the overall evolution of dynamical systems has been observed in, for example, stochastic bifurcation
\cite{Crauel, CarLanRob01, Horst}, stochastic resonance \cite{imkeller2002model},
 and  noise-induced pattern formation \cite{Gar, blomker2003pattern}.
Hence taking stochastic effects   into account is of
central importance for mathematical modeling of
complex systems under uncertainty.   It is therefore crucial to investigate dynamics under uncertainty, in the context of models arising from applications in, for example, geophysical and biophysical systems. 

Fluctuations in these complex systems are often \emph{non-Gaussian} (modeled by L\'evy motion $L_t$) \cite{Woy,Dit, Shlesinger,taqqu,dybiec2009levy}, rather than Gaussian (modeled, say, by Brownian motion $B_t$). \FredNote{Can we say briefly why, say from the application side?} This leads to a
 stochastic   differential equation (SDE)  \cite{Arnold,   Oksendal, DuanBook2015, Applebaum}
 driven by a L\'evy motion as well as Brownian motion:
 $$
 dX_t= f(X_t) dt + g(X_t) dB_t + \sigma(X_{t-}) dL_t.
 $$
 We often think this as the governing equation for random `particles'. A physically significant L\'evy motion is the  $\alpha$-stable L\'evy motion $L_t^\alpha$ (for $0<\alpha<2$).  
 When the noisy fluctuations are absent, we then have a   deterministic  dynamical system described by an ordinary differential equation (ODE): $ \frac{dx}{dt}=f(x).$
 
 
Stochastic dynamical systems are also extremely relevant to several data science applications such as machine learning in the presence of significant noise. Recently, ordinary and stochastic differential equations, parameterized by neural networks, have been used for learning regression and classification tasks for noisy irregular datasets \cite{chen2018neural,rubanova2019latent,jia2019neural,tzen2019neural,look2020deterministic,liu2019neural}. Specifically, SDEs, with non-Gaussian as well as Gaussian noise, have been used to represent the `flow' of information from the input of a deep learning framework to the output. The assumption of noise-based forcing of dynamics imparts significant robustness to the quality of learning and the ability to generalize to unseen data. The study of complex stochastic dynamical systems with a focus on their interface with state-of-the-art deep learning architectures is essential for multiplying opportunities for applied mathematics undergraduate and graduate students in data-driven modeling and simulation. 


\subsubsection*{Quantifying transition phenomena}
To understand transition   in    a stochastic system, we  study the     `ensemble' (or `continua') quantities: mean exit time (exiting from a dynamical regime),   escape probability (from one dynamical regime to another), and  the transition  probability density. \FredNote{How will this help us better understand the applications mentioned above?}
This leads to deterministic  nonlocal (integro-differential) equations for mean exit time and for escape probability, and a nonlocal Fokker-Planck equation,
in which the usual Laplace operator is replaced
by a nonlocal Laplace operator (which is (related to) the generator of L\'evy motion $L_t^\alpha$):  $-(-\Delta)^{\frac{\alpha}2  }$, for $\alpha \in (0, 2)$.
 
 
As the usual numerical methods (such as finite difference) are not feasible, we will develop a method to simulate these high dimensional nonlocal equations based on a stationary or temporal normalizing flow model which is a subset of generative machine learning models using deep learning. \FredNote{Are we going to use generative machine learning models to study SDEs with L\'evy noise that model deep learning models?} This method involves techniques to handle the nonlocal terms implicitly since arbitrary probability densities can be learned from samples alone. Subsequently, `ensemble' quantities, including the mean trajectories (when exist), are high dimensional integrals with respect to the probability density \cite{DuanBook2015}.  Thus we will validate our fit distributions with Monte Carlo simulations, in particular highly stratified or quasi-Monte Carlo variants, where our team has made recent theoretical advances \cite{Hic17a, HicEtal17a} and efficient implementations \cite{QMCPy2020a}. 

Furthermore, we will investigate the most probable transition pathway between dynamically different regimes (e.g., metastable states) during finite transition time. This pathway is the minimizer of  
of the associated Onsager-Machlup action functional, $\mathcal{A}(z, \dot z) = \int_0^T \mathcal{L}(z(t), \dot z(t)) dt,$ where the Lagrangian  $\mathcal{L}$ depends on the vector field $f$, noise intensities $g, \sigma$, and  jump measure $\nu$ for L\'evy motion; for background, see   our recent works \cite{ChaoDuanOM,HuangYF}. The   transition time $T$ may also be estimated \cite{HuangYF2020}. We will explore how the energetic variational approaches in Project 2 may benefit us in the minimization of the Onsager-Machlup action functional here. 


 
\subsubsection*{Extracting transition phenomena from noisy data}
Some   stochastic systems may be only known or partially known (where some mechanisms are missing) by   noisy observations or simulated data sets. We will  develop techniques to extract stochastic governing laws \cite{YangLi2020a} and then examine transition phenomena by computing mean exit time, escape probability and transition probability density as above. 
There are several potential interfaces with state-of-the-art generative machine learning here which has been constructing novel methods to learn probability densities from data as neural stochastic differential equations. Furthermore, our colleagues'  works on 
designing and optimizing sensor networks, and graph theory \cite{karwa2016statistical,Calines2008MonitoringSF} will be explored to guide smart data collection for interface with such novel algorithms. 

A class of high dimensional dynamic systems may be interpreted as networks, such as   gene regulation networks \cite{Raser2005} which are modeled by systems of stochastic differential equations \cite{Suel06}.  In a gene regulation network, a transition from low to high concentration of a promoter protein may indicate transcription \cite{Stefan,ZLDK}. Studying transitions in large networks is also of vital importance for resiliency analyses and control of powergrid infrastructure and for supply chain modeling in operations research which may be modeled as stochastic processes on graphs \cite{shin2020graph,anghel2007stochastic,nardelli2014models}. For large networks with multiple sources, sinks and connectivity measures of several variables, extremely high dimensional systems are commonly obtained. Furthermore, optimization of these graphs for desired quantities of interest require the solution of a complex mixed-integer nonlinear programming problem with constraints that may not be nominal input-output mappings but aforementioned complex dynamical systems \cite{shin2020decentralized,sampat2017optimization,kim2019graph,shin2021exponential}.


%%%%%%%%%%%%%%%%%%%%%%%%%%%%%%%%%%%%%%%%%%%%%%%%%%%%%%%%
\subsection*{Project 2. Energetic Variational Approaches for Complex Dynamics:}
%%%%%%%%%%%%%%%%%%%%%%%%%%%%%%%%%%%%%%%%%%%%%%%%%%%%%%%%
Chun Liu (leader), Yiwei Wang (Illinois Tech, Applied Math), and Georgia Papavasiliou (Illinois Tech, Biomedical Engineering).

\subsubsection*{Background} The research focus will be on the applications of the energetic variational approach in mathematical modeling, nonlinear partial differential equation, scientific computing and machine learning. The framework of energetic variational approach, originated from seminal works of Raleigh \cite{strutt1871some} and Onsager \cite{onsager1931reciprocal,onsager1931reciprocal2}, provides a paradigm to determine the dynamics of a non-equilibrium system from prescribed energy-dissipation laws.
These approaches have motivated much work in the area where it is crucial to preserve the underlying
thermodynamics structures, including quantification of noises and fluctuation of the systems.
The variational principle has been employed to study many complex fluids, such as liquid crystals, two-phase flows, and ionic fluids, which involve the coupling and competition of various mechanical and chemical mechanisms in different scales \cite{lin2001static, feng2005energetic, LiLiZh05, Lin2007, liu2009introduction, du2009energetic, sun2009energetic, eisenberg2010energy, Giga2017, Liu2019, knopf2020phase}.

In an isothermal closed system, an energy-dissipation law, coming from the first and second law of thermodynamics, is often given by $\frac{\dd}{\dd t} E^{\rm total} = - \triangle$,
where $E^{\rm total}$ is the total energy, including both the kinetic energy $\mathcal{K}$ and the Helmholtz free energy $\mathcal{F}$, and $\triangle \geq 0$ is the rate of the energy dissipation which is equal to the entropy production in the situation. The energy-dissipation law, along with the kinematics relation, %describe all the physics and the assumptions for a given non-equilibrium system. 
capture all specific physical and biological properties for a given non-equilibrium system. Starting with an energy-dissipation law, the EnVarA framework derives the dynamics of the systems through two variational principles, the Least Action Principle (LAP) and the Maximum Dissipation Principle (MDP). The LAP, which states the equation of motion for a Hamiltonian system
can be derived from the variation of the action functional $\mathcal{A} = \int_{0}^T [\mathcal{K} - \mathcal{F}] \dd t$ with respect to the flow maps, gives a unique procedure to derive the conservative force for the system. The LAP is indeed a manifestation of the rule $\delta E = {\rm force} \cdot \delta x$. The MDP, variation of the dissipation potential $\mathcal{D}$, which equals to $\frac{1}{2}\triangle$ in the linear response regime, with respect to the rate (such as velocity), gives the dissipation force for the system. In turn, the force balance condition leads to the evolution equation to the system, $\frac{\delta \mathcal{D}}{\delta \x_t} = \frac{\delta \mathcal{A}}{\delta \x}.$ Through different choices of the free energy $\mathcal{F}$ and the dissipation $\triangle$,
the EnVarA %can 
systematically deals with the coupling and competition of mechanical, chemical and thermal mechanisms in different time scales in a thermodynamically consistent framework. It 
provides an unified framework to study various
dynamical systems demonstrated in the other projects, and
moreover, gives a way to quantify energy transduction between different processes.  

\subsubsection*{Applications in Biology and Engineering}
The team has the track record and relevant expertise to lead the projects. We plan to focus on projects related to the modeling of active soft materials in biological environments. 
The unified framework will provide the insights for the systematical
study of coupling and competition of these multiscale and multiphysics systems.
The interdisciplinary nature of the team will provide an opportunity to have balanced training in different approaches of applied math, most notably multiscale modeling and analysis associated with the applications in biology and soft matter physics.

\subsubsection*{Physics Preserving Numerical Simulations}
Energetic variational formulation also provides guidelines to develop numerical schemes that preserves the basic physics, such as frame-indifference, positivity of physical quantities  and conservation of mass, for many complicated systems arising from multiscale multiphysics problem in physics, biology and material science \cite{liu2020variational, liu2020lagrangian, liu2020structure}. The same numerical frame can also be applied develop new algorithms for machine learning \cite{wang2020particle}. Different spatial discretizations, including finite element, spectral methods, particle methods and neural network discretization, and time-stepping techniques can be incorporated in the numerical framework. These projects will provides training in the areas of scientific computation, multi-scale simulation and machine learning.



%%%%%%%%%%%%%%%%%%%%%%%%%%%%%%%%%%%%%%%%%%%%%%%%%%%%%%%%
\subsection*{Project 3: Mathematical Tools for Quantum Dynamics:}
%%%%%%%%%%%%%%%%%%%%%%%%%%%%%%%%%%%%%%%%%%%%%%%%%%%%%%%%
Jinqiao Duan (leader), Fred Hickernell, Romit Maulik (Illinois Tech and Argonne National Lab), Carlo Segre (Illinois Tech Physics), and Martin Suchara (Argonne National Lab).

\subsubsection*{Background} The development of quantum science and its exploitation in technology is a topic of great interest and activity in many countries. The United States \cite{raymer2019us}, Europe \cite{riedel2019europe}, and China \cite{kania2018quantum} have invested heavily in long term ``quantum initiatives’’.  It has been stated that we are in the midst of the ``second quantum revolution’’ \cite{kania2018quantum}. Indeed,  most people tend not to realize just how much quantum mechanics influences their day-to-day life. For example, consider that standard household light bulb. Advances in light emitting diode (LED) technology have led to ``white’’ light bulbs that can now be purchased anywhere light bulbs are purchased. These LED light bulbs are more energy efficient, last longer, and are more environmentally friendly than light bulbs of the past. This is all thanks to quantum mechanics, which is fundamental to LED physics and technology. The impact of this rather mundane example is significant. But the impact of more complex physical and technological quantum advances, such as quantum computing, quantum cryptography, quantum sensors \cite{ng2020guest}, and even quantum artificial intelligence \cite{taylor2020quantum} will change the world and our way of life.

Therefore it is essential in education nowadays that students develop a level of ``quantum literacy’’ \cite{foti2021quantum} that will enable them to interpret and evaluate these new phenomena and technologies in a way that will enable them to assess their impact, not only on their own  lives, but on society and the world. Developing this quantum literacy in the context of mathematical education,    at both the undergraduate and graduate   levels,  is a goal of this RTG project.


Quantum mechanics is not typically emphasized in most mathematics programs.  Indeed, many of the unusual properties of quantum mechanics, such as tunneling, entanglement, superposition of multiple states, interference of states, uncertainty, and quantum measurement can be understood from the fundamental physical principle of ``wave-particle’’ duality and de Broglie’s relation between the momentum and the wavelength of a particle.  From this one is led to the Schr\"{o}dinger equation description of particles in terms of waves. In fact, the physical setting of quantum mechanics is described by a few (mathematical) axioms that describe the state space, observable quantities, measurement, and time evolution.  Once this physical setting is translated into mathematics, the implications and consequences of the mathematics is a natural setting for the participation of mathematicians. However, the required mathematical skills and knowledge are not typically emphasized or even covered in the relevant courses. A goal of this RTG is to remedy this situation, both through training undergraduate, graduate, and postdoctoral students, and through the development of a course appropriate for mathematicians, but which bridges this gap. This course will lead naturally into a variety of research areas that fit naturally into the context of the other research topics, that we will describe.

\subsubsection*{Exploring the Interplay Between Quantum Mechanical Uncertainty and Noise.} Uncertainty relations between quantum mechanical observables (such as position and momentum) arise naturally as a result of the mathematical formulation of quantum mechanics, in the absence of noise \cite{Griffiths2018IntroductionTQ}. This project will explore the interplay between quantum mechanical uncertainty and the influence of external noise \cite{Lindgren2019QuantumMC, Nagasawa2000StochasticPI} on the quantum mechanical system. In particular, we will examine how certain quantum dynamical behaviors  may be understood through stochastic minimization \cite{Lindgren2019QuantumMC} and affected by quantum noise \cite{Nurdin2019QuantumSP}. This may   benefit from our recent work on effective quantum wave factorization \cite{ZHANG2020132573}. This also relates to   Project 1 - Dynamics under Uncertainty.


\subsubsection*{Applying Methods of Lagrangian Fluid Transport to the Bohmian Trajectory Formulation of Quantum Mechanics.}   A `trajectory'  formulation of quantum mechanics  equivalent to the Schr\"{o}dinger equation can be given \cite{Bohmian, Holland1993TheQT} that emphasizes the flow of probability, or ``probability current’’ in space. 
In this formulation, the state of a system of $N$ quantum particles is described by its wave function $\psi=\psi(q_1,…,q_N)=\psi(q)$ via the Schr\"{o}dinger equation  on the space of possible configurations $q$ of the system, together with its actual configuration $Q$ defined by the actual positions $Q_1,...,Q_N$ of its particles. The trajectory of these quantum particles obey the dynamical system
$$
\frac{dQ_k}{dt}= \frac{\hbar}{m_k} \text{Im} \frac{\bar\psi  \partial_k \psi}{|\psi|^2}(Q_1,…,Q_N),\;\; k=1, ..., N,  $$
where $\hbar$ is the reduced Planck constant and $m_k$ is the mass of $k$-th particle.
This is the pilot-wave, or hydrodynamical, point of view. We will explore this point of view from the perspective of the dynamical systems approach to Lagrangian transport in fluid dynamics, including an examination of phase portraits of quantum dynamical trajectories \cite{Berndl1995OnTG} and transition phenomena \cite{waalkens2007wigner,Micha2006QuantumDW, Dittrich2016}. 


%%%%%%%%%%%%%%%%%%%%%%%%%%%%%%%%%%%%%%%%%%%%%%%%%%%%%%%%
\section{Recruitment and Retention Plan} \label{sec:RandR}
%%%%%%%%%%%%%%%%%%%%%%%%%%%%%%%%%%%%%%%%%%%%%%%%%%%%%%%%
This RTG involves BS and PhD students as well as postdocs, referred to collectively as trainees.  The PIs have a track record of mentoring these applied mathematicians at all stages.  In the past five years, we have supervised research for over a dozen undergraduate students, many of whom have published their research and entered PhD programs at top 50 universities. One faculty member also supervised research for a high school student.  The PIs have  mentored dozens of PhD students, some of whom have entered academic positions at research universities, and others whom have entered research related jobs in industry and national laboratories.  The PIs have also mentored a number of postdocs, who have been attracted to the activity of our research groups.

At present, our department has about \numUG BSs, \numPhD PhD students, and \numPostDoc postdocs.  This RTG aims to increase the numbers of students and postdocs entering applied mathematics research through Illinois Tech and to increase the diversity.  We will do this by growing our recruitment and retention efforts and providing enhanced research training.

\subsection*{BS Students}
Our Department of Applied Mathematics offers a BS in applied mathematics a variety of specializations as well as options for double majors and co-terminal joint BS-MS degrees. We also offer a BS in statistics.  Some of our majors enter as transfers from junior colleges and some transfer to our major or add our major as a double major after entering Illinois Tech.

Each year we will recruit 3--4 new BS applied mathematics or statistics students into the RTG, normally in the middle of their second year standing (as measured by credits obtained).  This does not necessarily correspond to their second year at Illinois Tech due to AP or transfer credits.

All majors must take the semester course MATH 100 Introduction to the Profession.  We will devote a session or two to research opportunities in the mathematical sciences and introduce our RTG program. As an aside, a Latina first year student who listened to Hickernell's presentation in MATH 100 several years back did undergraduate research under him as well as under another faculty member, received her BS/MS degree from Illinois Tech and is now enrolled in a PhD program at University of Illinois at Chicago.

To attract and retain talented and motivated students, we will provide opportunities for exploring mathematics outside the classroom through our active AWM and the SIAM student chapters.  Each year we will co-sponsor with these chapters a session introducing students to our RTG and advising them on how to apply.  The introductory session will feature existing trainees describing their experiences. It may coincide with one of our activities showcasing our research.

To increase the diversity and preparedness of the pool of undergraduate students, we will collaborate with our admissions office to reach out to the nearby high schools and junior colleges, which serve a higher proportion of underrepresented minority students.  We will also pitch our RTG to the high school students attending Illinois Tech's summer programs to attract them to apply to Illinois Tech as applied mathematics majors.

To retain our undergraduate trainees, we will assign each a faculty mentor as well as an experienced PhD student or postdoc mentor.  This will be in addition to the advising that our department already provides.  The mentors will meet individually with each BS student at least once per term to discuss progress, course selection, summer plans, and career plans.  The mentors will also make themselves available to help the trainees with coursework or research questions.  PI Pelsmajer served for many years as the Associate Chair and Director of Undergraduate Studies. He will coach our mentors on how to fulfill their responsibilities well.

\subsection*{PhD students} 
Our department has a single PhD program in applied mathematics with five emphases: applied analysis, discrete mathematics, computational mathematics, statistics, and stochastics.  Faculty strength covers these five areas and our research projects draw on strengths from all of them. Our success in recruiting domestic PhD students has been growing, including four within the last two years.

PhD students act as force multipliers in research programs of their advisors as well as undergraduate instructional support of the department. To increase the size, quality, and diversity of our PhD program, each year our RTG program will recruit about three new PhD students to be supported for the their first three years of study.  One avenue of increasing the size and diversity of our applicant pool, presently numbering 30--35 per year, will be through our summer REU programs for undergraduates described in Section \ref{sec:summer}.  The SURE program, in particular, involves partnerships with minority serving institutions.  During our summer 2021 SURE pilot, we have already engaged one African American student from an HBCU, who intends to pursue a PhD.  

The PIs will visit minority serving institutions to give seminar talks to students and faculty on mathematical topics of interest related to our RTG.  We will build professional relationships with faculty advisors so that they are confident to recommend their students to our PhD program.  Our PIs will also participate in events that attract underrepresented minorities, such as the annual SACNAS meeting, the AMS joint meeting, and AWM events.

Increasing our PhD stipends to make them comparable to those offered by other Chicagoland universities will help recruitment. To do this in a sustainable way, we will partner with Advancement to raise money for scholarship support for domestic PhD students, while advocating for higher stipends for all of our PhD students.

To retain our PhD students, we will assign them a faculty mentor from among the PIs when they first enter our program.  This mentor may change if the student's research interests are best served by a different mentor.  Each PhD student will meet once a semester with the mentor and our Associate Chair for Graduate
Studies to monitor progress towards completion of the PhD.  Selection of courses and the thesis committee
will be decided in conjunction with the faculty mentor, who will also be responsible in identifying
and recommending the student for suitable summer internships. When the PhD student is teaching, the course instructor will observe their teaching, provide feedback.

\subsection*{Postdoctoral fellows}
We will recruit postdoctoral fellows through our professional networks, approaching promising PhD students that we meet at conferences, and advertisements in \url{mathjobs.org}. Each postdoctoral fellow will have two faculty mentors. One mentor will center on research activities, such as preparation of 
publications, research presentations, mentoring students, and proposal writing. A second faculty member will mentor on teaching activities by way of visits and advising on suitable Center for Learning Innovation workshops they should attend. More details on mentoring are provided in the Postdoctoral Mentoring Plan.


%%%%%%%%%%%%%%%%%%%%%%%%%%%%%%%%%%%%%%%%%%%%%%%%%%%%%%%%
\section{Research Group Activities}
%%%%%%%%%%%%%%%%%%%%%%%%%%%%%%%%%%%%%%%%%%%%%%%%%%%%%%%%
The groups of faculty, postdocs, and students corresponding to the three research problems in Section \ref{sec:researchproblems} will meet regularly---weekly during the fall and spring semesters and more frequently during the summer.  These group meetings will build upon our existing Stochastic Dynamics Seminar for project 1, and Multiscale and Complex Fluids Seminar for project 2. A new Quantum Dynamics Seminar will be added for project 3.  Although not all postdocs and students involved in this RTG will attend all research groups, the times will be scheduled to not overlap. 

All trainees (students and postdocs) supported by this RTG will be assigned to one or two projects and join the corresponding seminar.  Trainees may enter or exit these research groups at different times of the year, e.g., some will participate during a summer only, while others will participate over the course of a couple of academic years. Thus, certain background lectures will need to be repeated periodically.  Initially these will be given by the PIs and our faculty and scientific collaborators, and the lectures will be recorded so that students and postdocs can refer to them.  However, part of our training will be to mentor our postdocs and experienced students to give these background lectures as well.  

Meetings will be held in hybrid mode, i.e., in person and via videoconference, to allow the greatest participation.  Workshops, tutorials, and public lectures will be broadcast via our Illinois Tech YouTube channel to allow greater access, both synchronously and asynchronously.

Trainees entering the research groups will be provided with a faculty mentor and a postdoc or experienced student mentor.  New trainees will be given an article to read related to the research problem and report on it to the seminar, describing what are the significant claims, pointing out drawbacks, and suggesting directions for extension. The articles may be in mathematical sciences or in an application area.

As trainees become engaged in research, they will report on the progress at the regular seminar meetings. They will report on significant progress at local and regional conferences.  Substantial progress will be reported at national and international conferences.

Each year, each research group will organize a two-day workshop, either at Illinois Tech or in collaboration with IMSI or an international center such as those in Banff.  The workshops will showcase our own results to others outside Illinois Tech and also let our trainees hear from international experts.

Members of research groups will actively apply for external funding to further support their research activities.  PhD students and postdoctoral fellows will be asked to draft parts of the proposals and also to read and comment on draft proposals as part of their training.  BS and PhD students will be encouraged to apply for graduate and postdoctoral fellowships from the NSF and other external funding agencies.

%%%%%%%%%%%%%%%%%%%%%%%%%%%%%%%%%%%%%%%%%%%%%%%%%%%%%%%%
\section{Curricular and Co-Curricular Training}
%%%%%%%%%%%%%%%%%%%%%%%%%%%%%%%%%%%%%%%%%%%%%%%%%%%%%%%%
To support their research activities, student trainees will take courses and be participate in co-curricular events common to all research groups.  PhD students and post-docs will be involved in teaching and mentoring. 

\subsection*{BS Students}

Beyond the calculus, introductory ordinary differential equations, and linear algebra sequence, undergraduates will be strongly urged to take MATH 350 Introduction to Computational Mathematics, MATH 475 Probability, and MATH 488 Ordinary Differential Equations and Dynamical Systems either before entering or early in their traineeships.  These course will cover important ideas needed in their research, and the first two are required for the applied mathematics major.  

MATH 350 covers a spectrum of computational methods, but so far has no coverage of random algorithms. An introduction to Monte Carlo methods will be added to MATH 350 to address this deficiency.

PI Duan has taught MATH 488 multiple  times in the past 20 years.  He will update the syllabus to integrate recent advances, such as computational and data science techniques for high dimensional dynamics. 

Trainees will be encouraged to enroll in MATH 491 Reading and Research for one or more semesters to obtain credit for research being done during the academic year.  Trainees will be welcomed into our own summer program, but will also be encouraged to spend another summer in an REU or internship elsewhere to gain a broader experience.

We will create a monthly RTG undergraduate evening event for our applied math majors---both potential and existing trainees.  In early fall, we will give a presentation describing the overall RTG program, with the goal of recruiting students. Subsequent meetings will include presentations from   career services, campus student groups that promote research and innovation, such as the Camras Scholars, Intinium, the Undergraduate Research Journal at IIT, and also presentations from Argonne and IMSI advertising their programs. 

By late fall, we will give brief presentations of the research topics of the RTG program, which will serve as an introduction of faculty members of the different
groups. Meetings in the the fall term and and winter will concentrate on training and helping students apply for PhD programs, REUs, and research related internships. 



\subsection*{PhD Students}

%The majority of our PhD recipients’ first jobs are in non-academic positions. We view our success in placing students in a broad range of organizations as a unique and positive indicator, which we want to maintain. The research topics and interdisciplinary approach should enhance opportunities for jobs in industry and national laboratories for our RTG students and at the same time increase the number of PhD recipients going into high quality postdoctoral positions.

We will enroll 3-4 new PhD trainees each year, and each trainee will be supported for up to three years, typically their first three year.  While supported by RTG traineeships, these PhD students will be released from other duties to concentrate on their coursework, research, and other RTG activities.  For example, bright students that are lacking some crucial mathematical background may take some upper division undergraduate courses.  

PhD student trainees will complete the foundational courses in analysis and computation required by our PhD program.  In addition, they will be urged to take one or more of the following courses.

\subsubsection*{MATH 544 Stochastic Dynamics}
We will revise this graduate course  incorporating topics on  high dimensional stochastic systems via data-driven algorithms.  The course will fill a gap by addressing simulation methods for nonlocal equations over high dimensional domains.

\subsubsection*{MATH 565 Monte Carlo Methods} Originally designed for mathematical finance majors, we will revise this course to include topics such simulation of (partial) differential equations with random coefficients.  The course already introduces quasi-Monte Carlo methods, a powerful alternative for speeding up expensive Monte Carlo calculations.

\subsubsection*{New Course: Introduction to Mathematical Theories of Complex Fluids} 
Complex fluids are ubiquitous in our daily life and important in many industrial, physical and biological applications. Studying these materials requires a wide range 
of tools and techniques for different disciplinary, even within mathematics. A new upper undergraduate/graduate course will be developed on mathematical theories of complex fluids. The course will be self contained and covers basic mechanics, non-equilibrium thermodynamics, multi-scale modeling and analysis with applications in engineering and biology. The course will fill the gaps between traditional mathematical courses and engineering courses.
 
\subsubsection*{New Course: Introduction to Mathematical Framework for Quantum Mechanics} We will build this course from  a   undergraduate course   taught by team member Carlo Segre, and another one taught by our Research Professor Stephen Wiggins (available on YouTube). PI Duan   taught a minicourse in Quantum Dynamics \cite{Gutzwiller1990ChaosIC, Holland1993TheQT,Lindgren2019QuantumMC,Micha2006QuantumDW} in 2019 and 2020 for graduate students and his collaborators.  We will invite our collaborators to co-teach this course. 
The YouTube format provides the opportunity to engage the students in an active learning format where they view the “short" lectures ahead of time and the in person class time  also includes  questions, discussions, and problem solving. Stephen Wiggins will co-teach this course with us and his letter of collaboration is a supplementary document.

By the end of Year 1, PhD trainees will enroll in one of the groups and be assigned a faculty mentor from the corresponding team. They will be encouraged to participate in the summer program and will receive one month support to work on a research project. \FredNote{Don't they have 12 month stipends already?  Why not have them help with the summer programs for undergraduates?} Year 2
will still be focused on completing necessary coursework.
Ideally, PhD students will do a research-related internship during the summer of Year 2 or Year 3. Students who do not have
a summer internship will do in-house research and participate in the RTG summer program
by giving lectures and mentoring BS students. By the end of Year 3, RTG students
should pass qualifying exams and present their PhD thesis proposal. They should also have participated in one   workshop or presentation run by the Center for Learning Innovation to prepare them for teaching experience. In Year 4, students will be fully engaged in their dissertation work straddling one or more of our research project areas. They will also teach a section of Calculus with the support of a teaching
mentor, who will observe a few classes and write a report. The summer of Year 4 and all of Year 5 should focus on completing their dissertation and applying for jobs and postdoctoral fellowships.

In total, PhD student trainees will have at least two semesters of either supervised teaching or research-related internship or both.  Internships may be at at Argonne or other government labs and or at data science companies, such as SigOpt (now a part of Intel) whose Head of Engineering, our alum Michael McCourt, has provided a letter  of collaboration.

\subsection*{Postdoctoral Fellows}
Postdoctoral fellows must become independent researchers with a well-defined research program who can identify interesting, accessible research problems, write research proposals and publications, and teach others.  To hone their teaching skills, postdocs will teach one course per semester. They will help lead the summer programs and research group meetings, prepare and give tutorials on background material, assist in giving minicourses, and help organize workshops.  In these activities they will be mentored by faculty.

\subsection*{Minicourses}
Our collaborators, at Argonne and other institutions, and some visitors will offer minicourses (about 10 lecture hours each) to supplement regular courses and to expose trainees to the frontiers of our research projects. Possible minicourses include  Machine Learning on Encrypted Data,  Non-equilibrium Statistical Mechanics,  Stochastic Hamiltonian Dynamics, Bohmian Quantum Dynamics, High Dimensional Numerical Analysis, and Foundation of Machine Learning, among others. \FredNote{When will they be offered and over what time span?}

\subsection*{Colloquium} The colloquium mentioned above, although aimed in part at BS students, will also have a number of activities aimed at PhD students and postdoctoral fellows as well.  Some of these are mentioned below.

\subsection*{Communication}
Communication is crucial to a scientist, and the crafting of the message depends on the audience:  research group, journal referees, proposal evaluation panel, students, and/or the general public.  Given the public's frequent misunderstanding of science, we will educate our trainees on how to reach them.  As part of our colloquium, during the winter and spring, C2ST (see the introduction) will run a workshop consisting of three modules of three hours each duration on communication to the public.  The topics will include speaking to the general public, social media engagement, writing for the general public.  This will be a continuation of the undergraduate RTG event mentioned above but include PhD students and postdoctoral fellows as well.

The RTG will provide travel support to RTG students and postdoctoral fellows to  present talks and posters and to network at conferences relevant to their research. We estimate for RTG students to receive support to one conference per year in the final 2 years of their studies and one conference per year to postdoctoral fellows.  In addition, trainees will have the opportunity to participate in local and regional activities such as the Chicago Area Undergraduate Research Symposium (CAURS), the Chicago Area SIAM Student Conference, SIAM Central Section meetings, Midwest Dynamical Systems Conferences, Midwest Numerical Analysis Day, and Midwest Probability Colloquium, and workshops sponsored by Argonne and IMSI. Trainees presenting will give a practice talk to their research group.

\subsection*{Career Coaching}
In every October, one or two colloquium sessions will cover applying for REUs, graduate schools, and jobs in academia and industry. These sessions will draw upon the experience of our faculty, experienced trainees, and alumni, such as Dr.\ Michael McCourt of SigOpt, an Intel company focusing on Bayesian optimization.  A yearly colloquium presentation  will be offered in September by one of the PI/Co-PIs or experts on writing research proposals. Faculty and postdoc mentors will comment on draft r\'esum\'es,  applications, and proposals.  


%\FredNote{Do we still want this?  We have a lot going on.} Led by RTG postdocs and graduate students, we will organize RTG  weekly   Colloquia ``Frontiers in Applied Mathematics",  aimed at fostering collaboration between all RTG teams and exposing new research advances to trainees and other members of our department and beyond.   Once a year, the RTG  will organize a Colloquium  under the umbrella of the Illinois Tech SIAM Student Chapter.  


%%%%%%%%%%%%%%%%%%%%%%%%%%%%%%%%%%%%%%%%%%%%%%
\section{Summer Program} \label{sec:summer}
%%%%%%%%%%%%%%%%%%%%%%%%%%%%%%%%%%%%%%%%%%%%%%

Summer is a great time for intense research effort when faculty and students alike take a break from formal coursework.  Postdoctoral fellows will help RTG faculty to lead our  summer programs.

%\subsection*{Undergraduate summer program.}
We will run an in-house ten-week undergraduate summer research experience.  The program will be targeted at rising third and fourth year BS students.  The topics chosen each year will come from one or two of the research projects in Section \label{sec:researchproblems}.  We will support eight students each year, with around two-thirds being from Illinois Tech and the remaining being from other institutions.  When recruiting students from other institutions, preference will be given to women and underrepresented minorities.  RTG funds
will pay a stipend and room and board to participating students.  This will allow students to engage with one another and their mentors outside of research time.  

The ten-week program will begin with tutorials given by postdoctoral fellows and experienced students introducing a few possible research topics.  Participants will do some background reading and choose their topic, either solo, or in a small group, within the first week or two.  Regular meetings will allow participants to present their progress, get feedback from the larger group, and hone their message for an end of program presentation. The program will be hosted at both Illinois Tech and Argonne.  In case of a COVID-19 type event, the summer program will be held online.

A faculty member will oversee each summer with a postdoctoral fellow and one or two PhD students helping mentor the students.  There will be a synergy between our summer program and the related research groups, which will continue to meet during the summer, although perhaps less regularly.

Summer Undergraduate Research Experience (SURE) is a program sponsored by the College of Computing, the Department of Applied
Mathematics and Department of Computer Science, which is aimed at underrepresented minority students in Chicagoland.  The program collaborates with local community colleges and Northeastern Illinois University, a minority serving institution.  Our summer program will collaborate with SURE to recruit and sponsor students interested in the research topics that we are offering.
 
 
%\subsection*{Graduate summer program} \FredNote{I do not understand this program and if or how it syncs with the undergraduate one.  Will our PhD students be pulled in two directions?} We will organize two-week summer programs for  graduate students, featuring (i) presentations by RTG faculty and collaborators about background and advances  in topics related to our research  projects; (ii) presentations by postdoctoral fellows and graduate students either about their research progress or describing specific articles related to the ongoing research projects. These will provide the general background information in the form accessible to undergraduate students.  A goal of   this summer program is to excite new students and inspire existing trainees in this RTG. 

%We plan to have this summer program to be held at Illinois Tech in Years 1, 3 and 5, and at Argonne National Lab in Years 2 and 4. 



\section{Outreach}

\subsection*{Socially Responsible Modeling, Computation, and Design
(SoReMo)}
This new initiative at Illinois Tech led by Sonja Petrovi\'c (Applied Math) advocates for ethical, equitable approaches in computation,
modeling, and design, contributing to the common good of Chicago and
beyond through research and education initiatives at Illinois Tech.
SoReMo holds the forum,  a public bimonthly seminar to help us advance
our mission by shaping conversations and providing guidance to our
student Fellows. Learn from international expert speakers from academia
and industry about cutting edge topics in the area of social responsibility.
SoReMo is also offering to support students with semester-long fellowships.   
With this proposal, we will incorporate the students with the projects that are relevant to aspects of social responsibility and could have impact to social justice, including those on the ocean-geography dynamic that is crucial in understanding the global warming, the biology system that is related to public health, the engineering applications on battery, fuel cell and clean energy.

\subsection*{Workshop on Diversity, Equity and Inclusion (DEI)}
Building on our collaboration with Argonne, we plan to co-organize this DEI workshop with the Division of Mathematics and Computer Science Division, whose director, Valerie Taylor, is an African American female, and Illinois Tech's future Vice President for DEI.  The workshop will consist of two parts: i) encouraging underrepresented groups to pursue STEM careers, and ii) teaching faculty, postdocs, and students how to support and advocate for underrepresented groups.

\subsection*{Public Lectures} We will organize yearly public lectures in collaboration with the Chicago Council on Science and Technology (C2ST) involving one or more of our local team.  The lectures will highlight the applications of our mathematics and excite our audience about the possibilities of mathematics.  

Once during the 5-year window, we will co-sponsor a distinguished public lecture  on a topic that falls within the theme of complex dynamical systems.  The distinguished visitor will come to Chicagoland for a few days, and in addition to the public lecture, also spend time meeting with our RTG faculty and trainees, give a research talk, and share his or her wisdom on being a successful applied mathematician. 
%A letter of collaboration from C2ST is uploaded as a supplementary document.  

\subsection*{Website and Newsletters} 
With the assistance of a student worker, we will establish a website for this RTG that serves both an inward facing and an outward facing function.  RTG trainees will help keep the news, events, publications, and bios current.  They will also edit a semi-annual electronic newsletter, 
featuring our education, research  and outreach  activities. 



%%%%%%%%%%%%%%%%%%%%%%%%%%%%%%%%%%%%%%%%%%%%%%%%%%%%%%%%
\section{Performance Assessment Plan  }
 
To ensure that we are fulfilling our commitments made in this proposal and making in course adjustments as needed, we will establish and advisory group of three senior academics and administrators, e.g., the Dean of Computing, the Dean of the Graduate College, and the Vice President for DEI.  The PIs will report to this advisory group on a semi-annual basis and take their advice on how to improve our program.

We will specifically collect and evaluate the following data on the following metrics:

\begin{description}
	\item[Participation] \phantom{a}
	\begin{itemize}
		\item Number of BS and PhD students participating in the monthly colloquia for (potential) trainees
		\item Number and demographics of those applying to and admitted as trainees
		\item Number and demographics of students applying to and admitted into the summer programs
	\end{itemize}
	\item[Outcomes] \phantom{a}
	\begin{itemize}
		\item Destinations (graduate programs, employment) of trainees and summer students
		\item Publications and presentations broken down by demographic
		\item Number of grant proposals submitted and awarded broken down by demographic
	\end{itemize}
	\item[Qualitative Feedback] \phantom{a}
\begin{itemize}
\item Feedback from participants in summer programs, workshops, etc.\ on the success of the program and suggested improvements for the next one
\item Feedback from trainees on the effectiveness of the mentors
\item Feedback from employers/supervisors of program alumni
\end{itemize}
\end{description}
We aim to see greater numbers and a more diverse pool of  applicants to our degree programs than at present.  We also aim to see more students following research careers than at present.  
 


%%%%%%%%%%%%%%%%%%%%%%%%%%%%%%%%%%%%%%%%%%%%%%%%%%%%%%%%
\section{Organization and Management Plan }
 
%\textbf{The management plan submitted in the proposal must contain a description of the actions that will be taken to achieve the goals set in the assessment plan. One basis for judging proposals will be the goals set and the likelihood that the actions described in the management plan will achieve them. }

%This section should also contain information on the plans to recruit and retain U.S. students and members of underrepresented groups. 
 
\subsubsection*{Management Plan:}
 The PI and 4 Co-PIs will oversee the running of the program, as RTG Director  and Co-Directors, respectively.  The tentative rotation has Duan to manage for Year 1, Hickernell for Year 2,    Liu for  Year 3,    Duan and Maulik for Year 4, and Duan and Pelsmajer for Year 5.   Together we     will oversee summer activities.  
The Director will oversee
the budget and assist in the running of RTG activities during the academic year. 

%For this to work effectively, the Director or Co-Directors will have a course release when they manage the program.

 \subsubsection*{Dissemination:} 
 We will create a   RTG website that will include a schedule of activities, publications and highlight
distinctions received by RTG participants. The website will link to other items of interest and a link will
be created for recruitment purposes. All presentations by RTG faculty and trainees  will    include a slide advertising our RTG.
Moreover, the resulting publications   will be on arxiv and journals, algorithms and software will be on GitHub, and workshop  presentations and  public lectures will be on our RTG YouTube channel.

 
 \subsubsection*{Post-RTG plan:}
With the support of the upper administration we will maintain the postdoctoral program with three
fellows each with a 1-1 teaching load. The teaching component for RTG graduate students represents
an increase in the number of classes taught by PhD students. Understanding the importance of having
some teaching experience on their resume, we will work with the upper administration to offer both RTG
stipends and teaching opportunities to all of our post-RTG PhD students. On the undergraduate front, we
will maintain numbers in terms of Camras scholars and BS students applying to graduate schools. Based on
our final assessment of the summer program, if we see merit, we will apply for a follow-up REU proposal.
For an overall assessment of the RTG we will keep contact for the next three years of all participants.


%%%%%%%%%%%%%%%%%%%%%%%%%%%%%%%%%%%%%%%%%%%%%%
%%%%%%%%%%%%%%%%%%%%%%%%%%%%%%%%%%%%%%%%%%%%%%
\section{Broader Impacts} 

%Although the basic undergraduate and graduate level mathematics curriculum provides sophisticated quantitative tools, it generally does not develop in students the \emph{mathematical maturity} to integrate these tools into a unified \emph{creative toolbox}. Developing this maturity is the overarching goal of this Research Training Grant (RTG).  
 

The overarching goal of this RTG is to increase US citizens, nationals and permanent residents in the workforce of  mathematical sciences. Especially, this RTG   contributes to  well-trained    diverse    workforce with mathematical maturity, quantum literacy,   data science techniques and communication skills.  The quantum literacy  and data science skills are   timely for the new  generation of mathematical scientists entering academia, industry and government.
   

This vertically integrated training program will involve women, native Americans and other members of underrepresented groups. Workshops, public lectures, learning innovation experiences, and research findings will be publicly available on the  RTG website, YouTube, GitHub, and academic journals. Written and oral communication skills will be built by requiring trainees to write to and speak to different audiences: referees for different journals, funding panels, and the general public. 

In addition to the   trainees, the entire Department of Applied Mathematics (both students and faculty) will   involve and benefit from this RTG in some capacity.    Students and junior faculty in Departments of Computer Science, Physics, and Biomedical Engineering will also be positively influenced by the innovative courses (e.g., Quantum Dynamics, Machine Learning on Encrypted Data) and the conducive research environment. 
 
This RTG trainees contribute to and benefit from collaborative activities with the NSF Institute for Mathematical and Statistical Innovation  and Argonne National Laboratory. 

This RTG enhances the public’s understanding and appreciation of mathematical sciences     and their impact on our society, with partnership of The Chicago Council on Science and Technology. 
 
It is expected that this RTG will serve as an inspiring model for STEM  training programs with partnerships of universities, government labs, industry, and nonprofit organizations.

 
 %%%%%%%%%%%%%%%%%%%%%%%%%%%%%%%%%%%%%%%%%%% 
\section{Results from Prior NSF Support}

%\noindent\underline{{\bf PI Duan:\,}} Duan and Xiaofan Li received prior support from: {\it DMS-1620449, 09/15/2016 - 08/31/2020, \$210,000. Theoretical and Numerical Studies of Nonlocal Equations Derived from Stochastic Differential Equations (SDEs) with L\'evy Noises}. Intellectual merit: This project develops efficient numerical techniques for investigating the macroscopic quantities that can help understand the dynamics of SDEs with $\alpha$-stable L\'evy noises, publications include \cite{Chen, WuFuDuan2019, YangDuan2020,ZhangZhuanDuan,WCDKL,CAI2019166,Cai_2017,IJNAM-17-151}. Broader impacts: One African American female Ph.D. student Julienne Kabre, and three other PhD students were supported  by the project. 
 
  
 
\subsection*{Prior Support of Duan} NSF grant  DMS-1620449, Theoretical and Numerical Studies of Nonlocal Equations Derived from Stochastic Differential Equations with L\'evy Noise, \$210,000; 09/15/2016 - 08/31/2020. Publications Resulting from this grant:
 \cite{ChenWu, ChenXL2020,  DannyTesfay,GaoTing2016, Gao2016,    Liu2019LvyNI, Lv2016OnAS, QiaoDuan2018,Wang2018NumericalAF, YangDuanWiggins2020,ZhangZhuanDuan,ZhengDuan2017,ZhengYY2020}.

 
\subsubsection*{Intellectual Merit}
(i)  \textbf{Numerical Methods}:
    We have developed a convergent and fast algorithm to compute macroscopic quantities from stochastic differential equations  with a-stable L\'evy noises. The numerical results on the mean   exit time, escape probability,  and the probability densities have been obtained \cite{ChenXL2020, GaoTing2016, Gao2016,     Wang2018NumericalAF}.
(ii)  \textbf{Theoretical Analysis}:
We have obtained results on  boundary regularity, interior regularity and maximal principles for   nonlocal partial differential equations for the mean   exit time,   escape probability and Fokker-Planck equation for stochastic dynamical systems with L\'evy noise. This has been inspired also by numerical scheme stability requirement of these nonlocal equations. We have further established a  low-dimensional reduction for a slow-fast data assimilation system with non-Gaussian a-stable L\'evy noise (when only slow variables are observable) via stochastic averaging and slow manifolds \cite{Lv2016OnAS,QiaoDuan2018,   ZhangZhuanDuan,ZhengDuan2017,ZhengYY2020}.
(iii)   \textbf{Applications}:
   We have revealed certain  dynamical features  for  the evolution of concentration in a genetic regulation model \cite{ChenWu} and a
   FitzHugh–Nagumo   neuronal model,   under additive and multiplicative     L\'evy fluctuations \cite{Liu2019LvyNI}, by examining and computing mean exit time, escape probability, and maximal likely trajectories (based on nonlocal Fokker-Planck equations for the stochastic systems).


 \subsubsection*{Broader Impacts}
 One African American female Ph.D. student, Julienne Kabre, was supported as full-time RA by the project. Three international students were supported during the summer of 2017 by the project.    Another Ph.D. student Senbao Jiang (from Courant Inst-NYU)  has been supported as research assistant under the grant. 
 
 %He has started to develop the numerical algorithm for solving the two-dimensional integro-differential equation induced by the $\alpha$-stable L\'evy noise.
 

\subsection*{Prior Support for Liu} NSF grants: DMS-1714401,  Topics in Complex Fluids and Biophysiology: the Energetic Variational Approaches, \$349,934, 7/1/2017 - 6/31/2021 and 
{NSF DMS-1950868, Collaborative Research: Multi-Scale Modeling and Numerical Methods for Charge Transport in Ion Channels, \$160,000.00, 8/15/2020 - 7/31/2023.}

 \subsubsection*{Intellectual merit.}  
% Chun Liu is the PI on the NSF grants:
% DMS-1412005, Energetic Variational Approaches in Complex Fluids and Electrophysiology, \$335,000.00,  8/1/2014 - 7/31/2017.
%grant #2024246 from the United States–Israel Binational Science Foundation (BSF), Israel .
PI Liu's research work during the last funding period was focused on the mathematical problems
arising from the modeling of the elastic complex fluids, together
with their physical, biological and engineering applications.
During that period, the PI had more than 40 peer-reviewed papers published in both mathematics and  other disciplinary journals. A general energetic variational framework had been established for various materials
involving multi-physic and multi-scale phenomenon,  in particular, viscoelasticity and polymers;  ionic fluids and general diffusions.
Some of the accomplishments include: (1) Application of the unified energetic variational framework for a wide class of complex fluids, including ion transport, multi-phase flow,
multi-component flow as well as systems involving moving  surfaces \cite{HuLiLi18,
yang_thermodynamically_2018,benesova_existence_2018, deng_largest_2017,xu_strong_2017,benesova_existence_2018,liu_energetic_2019, Kirshtein2020}, the non-isothermal effects \cite{de2019non,liu2018non, hsieh2020global, Jan-Eric}
and chemical reactions \cite{terebus2018discrete, wang2020field}. 
(2) Papers \cite{liu_energetic_2019, epshteyn2019large, Jan-Eric, hsieh2020global} presenting the analysis of the well-posedness, stability and singularity
for those dynamic systems. (3) Construction of corresponding stable and energy preserving numerical schemes \cite{duan_numerical_2019,xu_numerical_2019,duan_numerical_2019-2, wu2019energetic, liu2020lagrangian, liu2020variational, liu2020structure,duan2020structure}.
 (4)  Application of energetic variational approach to dynamics of ion channel and cell motions \cite{horng_continuum_2019,gavish_bistable_2018, liu_generalized_2017},
and related analytical studies \cite{WaLiTa17, hsieh2020global} with incorporation of thermal effects into these approaches \cite{liu2018non, wu2019energetic, hsieh2020global}.  (5) Development of new coarse grain  methods, including moment closure and projections, to study models involving multiple time scales
\cite{ma_fluctuation-dissipation_2017,ma_coarse-graining_2019}.

% [\color{red} I just noticed that some publication before 2017]}

\subsubsection*{Broader impacts}  Liu worked with
3 postdocs and 3 graduate students, along with several undergraduate students 
on related projects of this proposal. 
% The group also involves  s.
The students learned mathematical, physical and biological
aspects of the theory, while actively collaborating  with researchers in those fields.
They also attended and presented their work in workshops and conferences.


%%%%%%%%%%%%%%%%% 
\subsection*{Prior Support for Hickernell} NSF grant DMS-1522687, Stable, Efficient, Adaptive Algorithms for Approximation and Integration,
		\$270,000, 8/1/2015 -- 7/31/2018 along with Gregory E.\ Fasshauer (co-PI) and  Sou-Cheng Terrya Choi (female, Senior Personnel).  Other major contributors were Hickernell's research students: 
		six PhDs earned (two female), one current PhD student, and three MSs earned (two female).
Articles, theses, software, and preprints supported in
part by this grant include
\cite{ala_augmented_2017,
	ChoEtal17a,
	ChoEtal20a,
	Din15a,
	DinHic20a,
	GilEtal16a,
	Hic17a,
	HicJag18b,
	HicJim16a,
	HicEtal18a,
	HicEtal17a,
	HicKriWoz19a,
	RatHic19a,
	GilJim16b,
	JimHic16a,
	JohFasHic18a,
	Li16a,
	Liu17a,
	MarEtal18a,
	mccourt_stable_2017,
	MCCEtal19a,
	mishra_hybrid_2018,
	MisEtal19a,
	rashidinia_stable_2016,
	rashidinia_stable_2018,
	Zha18a,
	Zha17a,
	Zho15a,
	ZhoHic15a}.

%%%%%%%%%%%%%%%%%%%%%%%%%%%%%%%%%%%%%%%%%%%%%%%%%%%%%%%%%%%%%%%%%%%%%%%%%%%%%%%%%%%
\subsubsection*{Intellectual Merit}
Hickernell and collaborators developed adaptive algorithms for univariate integration, function approximation, and optimization \cite{ChoEtal17a,HicEtal14b, Din15a, Ton14a, Zha18a}.
Hickernell and collaborators developed globally adaptive algorithms for approximating multivariate integrals over the unit cube based on low discrepancy sequences \cite{HicJim16a,HicEtal17a,JimHic16a,RatHic19a}.  The error bounds underlying these adaptive cubatures rely on the Fourier coefficients of the sampled function values, either by tracking their decay rate or by using them to construct Bayesian credible intervals. The cost to bound the error using function data is essentially the same as order the cost to approximate the integral. 
Hickernell and collaborators investigated function approximation problems for Banach spaces defined by series representations \cite{DinHic20a,DinEtal20a}.  For all of these adaptive algorithms, defining a suitable cone of functions to be integrated, approximated, or optimized is key

\subsubsection*{Broader Impacts.}
Publications arising from this project are reference above.  
The project personnel spoke at many academic conferences and gave colloquium/seminar talks to mathematics and statistics departments and at several conferences.  Hickernell co-organized the 2016 Spring Research Conference and was a program leader for the SAMSI 2017--18 Quasi-Monte Carlo (QMC) Program.   He received the 2016 Joseph F.\ Traub Prize for Achievement in Information-Based Complexity. The algorithms from this research have been implemented in our open source MATLAB Guaranteed Automatic Integration Library (GAIL), which is used in the yearly graduate course in Monte Carlo methods.  Hickernell and the other senior personnel have mentored a number of research students associated with this project, including several female students.  More than a dozen undergraduates have been mentored and several have proceeded to graduate study in the mathematical sciences. 
 
\subsection*{Maulik and Pelsmajer} No current NSF funding or awards with an end date within the past five years. 

% \medskip 
 
%\noindent  \textbf{}: No current NSF funding or   awards with an end data within the past five years. 




\newpage
\pagenumbering{arabic}
\renewcommand{\thepage} {\arabic{page}}

\bibliographystyle{abbrv}
% \bibliographystyle{plain}
%\bibliographystyle{natbib}
\bibliography{asi,ref_art,stochas,ml_ref,datasci, VD, quantum, FJHown23, FJH23}

\end{document}    % End %

%%%%%%%%%%%%%%%%%%%%%%%%%%%%%%%%%%%%%%%%%%%%%%%%%%%%%%%
%%%%%%%%%%%%%%%%%%%%%%%%%%%%%%%%%%%%%%%%%%%%%%%%%%%%%%%
\section{Material Deleted from above is given below}
%%%%%%%%%%%%%%%%%%%%%%%%%%%%%%%%%%%%%%%%%%%%%%%%%%%%%%%
%%%%%%%%%%%%%%%%%%%%%%%%%%%%%%%%%%%%%%%%%%%%%%%%%%%%%%%


% Talk about this later:  In fact, IMSI has encouraged us to propose a joint  workshop relevant to our RTG theme,   thus helping publicize and recruit students and postdocs for this RTG.

%Other entities at Illinois Tech relevant to this RTG include the  Center for Interdisciplinary Scientific Computation,   Laboratory for Stochastic Dynamics and Computation, Center for Learning Innovation, and Career Services. The Camras Scholars Program help recruits and retain talented undergraduate students. Our Interprofessional Projects (IPRO) Program provides the alternative to a traditional undergraduate education. Our signature IPRO Program remains one of just a few programs of its kind in the country. IPRO joins students from various majors to work together to solve real-world problems, often on behalf of sponsor companies and nonprofits. A required academic program, IPRO teaches leadership, creativity, teamwork, design thinking, and project management—uniquely preparing students to succeed in a professional work environment. Moreover, the  Pritzker Institute of Biomedical Science and Engineering provides a limited number of research stipends for undergraduate math majors to conduct research in biophysical modeling. 
   
%All of our 18 tenure-stream faculty are actively engaged in research and teaching. 

%All faculty are fully engaged in teaching both undergraduate and graduate studies.   Current faculty members (Names?) have received the highest teaching recognition:.... Distinguished Teaching Professor Award. The PI Duan once received a Teaching Innovation Award. We also have four postdocs or visiting assistant professors with significant teaching responsibilities. 

%With this in mind, it is not unrealistic to expect that in one form or another, all Math faculty will participate in this RTG program, either as research team co-leaders, participants, or teaching mentors. In the next section and in the budget, we will identify those colleagues who will carry out specific RTG related tasks.

%We are in an ideal location    for research  training activities. For example, we hosted 3 CBMS-NSF research conferences and attracted .... 




  

\iffalse 
\subsection*{Illinois Tech resources  for this RTG}
Our university has the following resources to support this RTG.

 $\bullet$ Center for Interdisciplinary Scientific Computation
 
 $\bullet$ Laboratory for Stochastic Dynamics and Computation
 
 $\bullet$ Pritzker Institute of Biomedical Science and Engineering
 
  $\bullet$  The Camras Scholars Program
  
  This Program help recruits and retain talented undergraduate students. 
  
  $\bullet$  Center for Learning Innovation 
 
 
 $\bullet$   Career Services
 \fi
 
% Argonne National Laboratory resources ....
 
 
% 1. Introduce with connections for stochastic dynamical systems and deep learning and add literature review. 
% 2. Justify for RTG.


% Recent literature in deep learning have proposed the development of novel architectures that interpret the various hidden layers of feedforward neural networks as the discrete time steps of an ordinary differential equation. Neural ordinary differential equations \cite{chen2018neural,rubanova2019latent}   consider the continuum limit of neural networks where the input-output mapping is realized by solving a system of ordinary differential equations.
% %\begin{equation*}
% %    \frac{d \mathbf{u}}{d t} = f(\mathbf{u},t,\theta), \quad %\mathbf{u}(t_0) = \mathbf{u}_0, \quad t \in [t_0, t_F]
% %    \label{eq:node}
% %\end{equation*}
% %where $\mathbf{u}_0 \in \mathbb{R}^D$ is a feature vector for the input data, $\theta \in \mathbb{R}^N_p$ are the weights of the neural ODEs, $f: \mathbb{R}^D \times \mathbb{R} \times \mathbb{R}^N_p  \rightarrow \mathbb{R}^D$ is a neural network.
% %Essentially, the input-output mapping is learned by approximating %the vector field $f(u,t)$ through a data-driven approach. 
% This continuous model makes neural ODEs particularly attractive for learning and modelling the nonlinear dynamics of complex systems. They have been successfully incorporated into many data-driven models and recent literature has delved deeper into the connection between neural architectures and stochastic dynamical systems \cite{}. Therefore, the study of data science problems that can be framed as the solution of complex dynamical systems is essential for multiplying opportunities for applied mathematics undergraduate and graduate students in data-driven modeling and simulation. 
 
% and Riemannian geometry \cite{lou2020neural,falorsi2020neural} 

%For instance, it has been argued that diffusion by geophysical turbulence \cite{Shlesinger} corresponds  to a series of  ``pauses", when the particle is trapped by a coherent structure, and ``flights" or ``jumps" or other extreme events, when the particle moves in the jet flow. Paleoclimatic data \cite{Dit} also indicate such irregular processes. There are also experimental demonstrations of L\'evy flights  in optimal foraging theory and rapid geographical spread of emergent infectious disease.   Humphries {\it et. al.} \cite{Humphries}   used GPS to track the wandering black bowed albatrosses around an Island in Southern Indian Ocean to study the movement patterns of searching food.   They found that by fitting the data of the movement steps, the movement patterns obeys the power-law property with power parameter $\alpha=1.25$. To get the data set of human mobility that covers all length scales,  Brockmann   \cite{Brockmann}  collected data by online bill trackers, which   give successive spatial-temporal trajectory with a very high resolution. When fitting the data of probability of bill traveling at certain distances within a short period of time (less than one week), they found power-law distribution property with power parameter $\alpha=1.6$.

%This parameter is of great importance since by using it into the classic SIS model, they found probability density function patterns generated by non-local dispersal:  $\alpha-$stable L\'evy motions are strikingly similar to practical data of human influenza. A further example is a thermally activated motion of a test particle along a polymer,   shown in \cite{sokolov1997paradoxal, brockmann2002levy} to be subject to $\frac{1}{2}$-stable L\'evy motion due to polymer's self-intersections.  This motivates the investigation of dynamical systems driven by both Gaussian and non-Gaussian   fluctuations, especially the heavy-tailed,  $\alpha-$stable   L\'evy motions.  

%Both mean exit time and escape probability are described by nonlocal `elliptic' partial differential equations, with unusual boundary conditions corresponding to absorbing or escaping situations at microscopic levels. The probability density function for the solution of an SDE with  with L\'evy noise is described by a nonlocal Fokker-Planck equation.
 
%between distinguished states \cite{DaiMinChaos,LuYB2020, ZhengDuan2017, HuangYF, ZhengYY2020}.  and the Onsager-Machlup action functionals \cite{ChaoDuanOM, HuangYF2020}.  We will combine our group member' research in networks and    graph theory,  to reveal dynamical transitions between distinguished states (e.g., metastable states) of these large networks.

%Machine learning for high dimensional stochastic dynamics using deterministic quantities or indexes which are defined as mean or high dimensional integrals.

\iffalse
\noindent \textbf{Regular group activities:} We will hold   the weekly Stochastic Dynamics Seminar. The key component     is for     students present articles relevant for the main topics of the
project followed by a discussion of how the methods developed in our research could be applied to emerging
areas of   research (e.g., gene networks, climate models, machine learning inspired by stochastic dynamics). This will enhance our students’ grasp of the important current research topics
and potentially lead to new collaborations. We also propose that graduate students and postdoc regularly
make short presentations on their research progress to all RTG participants, followed by a discussion of next steps in each individual research project.


 \textbf{Yearly workshop:} A two-day workshop will be organized in the end of each academic year. Each workshop will consist   of presentations by graduate
students and postdocs, with one or two invited lectures (funded internally by the math department) given
by top researchers. These researchers will be chosen from experts.   
\fi

%Noise The numerical scheme .. can capture the coupling and the compeation, Besides,   the algorithm can also have applicaion in machine learning data science 


\iffalse
1. General energetic variational framework for complex fluids: 
      a) least action principle and maximum dissipation principle. 
      b) Navier-Stokes equations and elasticity. 
      c) viscoelastic materials: nonlinear elasticity, incompressible elasticity, viscoelasticity. 
      d) generalized diffusion, nonlocal diffusion.

2. Free interface motion in the mixture of different fluids: 
     a) conventional description: sharp interface description, water wave, vortex sheet, surface tensions. 
     b) diffusive interface description: microscopic background (self-consistent field theory), Flory-Higgins theory, 
         sharp interface limit, dynamics. 
     c) slippery boundary conditions and systems on (or near) surfaces. 
     d) Helfrich elastic bending energy and application to vesicle membranes. 

3. Multiscale modeling and analysis: 
     a) basis of stochastic differential equations: Fokker-Planck equations, diffusion, 
         Smoluchowski coagulation equations, variational formulations, kinetic theory. 
     b) micro-macro models for polymeric materials. 
     c) moment closure methods, Mori-Zwanzig formulation and other coarse grain methods. 

4. Ionic fluids and ion channels: 
     a) electroeheological (ER) fluids: Poisson-Boltzman fluids, like charge attaction (LCA) and charge inversion, 
         steric effects of ion particles, density function theory of Rosenfeld, equation of states. 
     b) basic physiology of ion channels and protein structures. 
     c) ionic fluids in ion channels. 
     d) ionic osmosis in biological systems. 

\noindent {\bf Regular group activities}: {We will continue the weekly Multiscale and Complex Fluids Seminar. One key activity is to have students present articles relevant for the main topics of the project and explore the potential applications. This could also foster new collaborations between the students and other researchers. We also propose that graduate students and postdoc regularly make short presentations on their research progress and discuss the possible new approaches.}

\noindent {\bf Yearly workshop}:  A two-day workshop will be organized either in Illinois Tech. We will also pursue opportunities for collaboration with international centers such as Banff International Research Station and the NSF Institute for Mathematical and Statistical Innovation (IMSI).
\fi


%\smallskip
%\noindent{\bf Simple Models Describing the Phase Space Formulation of Quantum Mechanics.} The potential well problems in the first two projects are formulated and solved in the standard Schr\"{o}dinger equation setting.  This project will explore reformulating those problems in the phase space setting using the Wigner function approach. The two approaches will be compared and contrasted.

%This is related to the ‘’Lagrangian transport’’ project.

%\smallskip

%\noindent
%{\bf Explorations of the Role of Bifurcations of Equilibria in Hamiltonian Systems on Classical Reaction Dynamics.} For two degree-of-freedom Hamiltonian systems the saddle-node, pitchfork, and Hamiltonian-Hopf  (or the (‘’Meyer-Schmidt’’) bifurcations typically occur.  We will explore how the normal forms for these bifurcations occur in chemical reaction dynamics and their influence on reaction.
  
%\item[Explorations of the Role of Bifurcations of Equilibria in Hamiltonian Systems on Quantum Reaction Dynamics.] This project will consider the classical normal forms of the previous project and the quantum mechanical manifestations of these bifurcations in reaction dynamics.

\smallskip

%Our training and research in this project will use the open source software QuTiP \url{http://qutip.org} and Qiskit \url{https://qiskit.org}, which are now standard tools for modelling and simulation of quantum phenomena.

\iffalse 

\noindent \textbf{Regular group activities:} We will hold   the weekly Quantum Dynamics Seminar.      Trainees in this team will present articles relevant for the main topics of the
project followed by a discussion of    next steps in   individual research issues.


 

\noindent \textbf{Yearly workshop:} A two-day workshop will be organized in the end of each academic year.  It will consist   of presentations by graduate
students and postdocs, with several invited lectures 
by top researchers.  
\fi

\iffalse
Our Interprofessional Projects (IPRO) Program provides the alternative to a traditional undergraduate education. Our signature IPRO Program remains one of just a few programs of its kind in the country. IPRO joins students from various majors to work together to solve real-world problems, often on behalf of sponsor companies and nonprofits. A required academic program, IPRO teaches leadership, creativity, teamwork, design thinking, and project management, uniquely preparing students to succeed in a professional work environment. Our department offers a recurring IPRO equivalent (an ``I Course'', taught as MATH 497) which has a special focus on techniques from mathematics and statistics.  Moreover, the  Pritzker Institute of Biomedical Science and Engineering provides a limited number of research stipends for undergraduate math majors to conduct research in biophysical modeling. 
\fi



Throughout the academic year, there
will be research presentations by RTG participants, mostly graduate students and postdoctoral fellows, with
more emphasis during the spring term. This last component serves two main purposes: to expose undergraduate
students to research and to prepare students for the four-week RTG summer program. The main
objective of this activity will be to increase the number of math and statistics majors that apply to PhD programs.
RTG funds will be used to provide food for these events, materials for presentations and advertising.



%\smallskip
%\noindent
%{\bf Linear Algebra from the Quantum Point of View.} It cannot be emphasized enough that Linear Algebra (Linear Space Theory in general) is fundamental to many areas of pure, applied, and computational mathematics, and it is also fundamental to the structure of quantum mechanics. Of course, ‘’every’’ mathematics department has courses in Linear Algebra. However, topics important to quantum mechanics are often omitted.  Our course will fill that gap. The starting  point will be to emphasize that the basic structure of quantum mechanics is complex in nature—the solution of Schr\"{o}dinger’s equation is complex valued and linear spaces are complex vector spaces. The language that we will use is Dirac notation.  This complex structure has implications for the nature of inner products, and this plays a key role in the mathematical structure of quantum mechanics.  Multi-particle quantum systems are essential for the notion of entanglement. In classical mechanics multi-particle systems are described by the cartesian product of state spaces for each particle. In quantum mechanics they are described by a tensor product, and this algebraic structure tends not to be treated in traditional linear algebra courses, but it is essential for describing the basic elements, and showing the advantages of, quantum computation, and this mathematical structure will be emphasized in our course.  The Schmidt decomposition provides an insightful characterization of entanglement based on tensor product spaces and its proof relies on the singular value decomposition, another topic of great importance (also in data science) that is typically neglected in a first linear algebra course, but it will be emphasized in our course.




%We envision one lecture to focus on developing models for physical systems and the other lecture to be primarily addressing the basic concepts behind the numerical methods used for simulations of such systems. For the rest of the two weeks, the focus will be on three sets of activities: (i) seminar-style presentations prepared by graduate students and postdoc, describing specific topics related to the ongoing research projects; (ii) undergraduate students working in teams on small research projects chosen in the beginning in the first week under the guidance of RTG faculty; (iii) visiting   Argonne National Laboratory and Pritzker Institute for Biomedical Science and Engineering and interacting with the faculty members, students, and postdocs conducting experimental works, with the objective of getting better understanding of the physical systems being modeled in the framework of the project. Each team of the undergraduate students will make a short presentation of the results of their work in the end of the second week.
 
% and minicourses on Data Science for Dynamical Systems, Advances in Multiscale Modeling, and Introduction to Quantum Dynamics offered by distinguished RTG visitors. 
 





%%%%%%%%%%%%%%%%%%%%%%%%%%%%%%%%%%%%%%%%%%%%%%%%%%%%%%%
%%%%%%%%%%%%%%%%%%%%%%%%%%%%%%%%%%%%%%%%%%%%%%%%%%%%%%%
%%%%% Material Deleted from above is given above
%%%%%%%%%%%%%%%%%%%%%%%%%%%%%%%%%%%%%%%%%%%%%%%%%%%%%%%
%%%%%%%%%%%%%%%%%%%%%%%%%%%%%%%%%%%%%%%%%%%%%%%%%%%%%%%













%%%%%%%%%%%%%%%%%%%%%%%%%%%%%%%
%%%%%%%%%%%%%%%%%%%%%%%%%%%%%
%%%%%%%%%%%%%%%
%%%%%%%%%%%%%%%%%%%%%%%%%%%%%%%%%%%%%%%%%%
\section{Supplementary Documentation}
 

\subsection{a. Letters of Collaboration}

--- from Argonne Lab


Signed letters of collaboration by the institution and other sources in support of the project should be included. If industrial or government laboratory internships are planned, letters indicating the willingness of the external organization and of individual external mentors (if known) to participate should also be included. These documents should be scanned and uploaded into the supplementary documentation section.
The letters of collaboration are meant to explain how the institution and the collaborating sites will provide an environment that supports the proposed research and training activities. It is acceptable for a letter of collaboration to briefly mention specific activities supported by the collaboration and listed in the proposal; however, each letter is limited to one page. Letters of recommendation or endorsement are not allowed.

\subsection{b. Trainee Data} 

All applicants are strongly encouraged to supply the following data. Note that data is requested for the group submitting the proposal, not for the entire department. For new RTG proposals, data should be included for the past five years. For a renewal of an existing RTG grant, data should be included for the past ten years.
A list of Ph.D. recipients, along with each individual's baccalaureate institution, time-to-degree, post-Ph.D. placement, and thesis advisor.
A list of postdoctoral associates (including holders of named instructorships and 2- or 3-year terminal assistant professors), their Ph.D. institutions, postdoctoral mentors, and post-appointment placements.
The dollar amount of funding by federal agencies for Research Experiences for Undergraduates (REUs), graduate students, and postdoctoral associates.





\subsubsection{A list of graduate students}

Romit:

Summer graduate students through Argonne:
Suraj Pawar, Oklahoma State University, Research Aide, 2020
Dominic Skinner, MIT, NSF MSGI, 2020
Alec Linot, University of Wisconsin Madison, Wallace Givens Associate, 2021
Sahil Bhola, University of Michigan, Research aide, 2021

\subsubsection{A list of postdoctoral associates}
  
  (including holders of named
     instructorships and 2- or 3-year terminal assistant professors),
     their Ph.D. institutions, postdoctoral mentors, and post-appointment
     placements.

     I know those from 2017. Can you check and keep it as a record?

     1) From Duan's group
     
     Jessica Xia 
     
     
     2) From Chun’s group:

     Stefan Metzger: 8/2017 — 8/2018, FAU ( Friedrich-Alexander
     University, Erlangen-Nurnberg), FAU
     Qing Cheng: 1/2019 — 1/2021, Xiamen University, Purdue University
     Yiwei Wang: 8/2018 — present, Peking University, IIT

     2) From Igor’s group

     Hyun-Jung Kim, 8/2018  — 8/2020, USC, UCSB

     3) From Sonja’s group

     Sara Jamshidi: 8/2018 — 8/2020, Penn State, Lake Forest College

     4) From Shuwang’s group

     Pedro Anjos: 8/2019 — present, Federal University of Pernambuco, IIT.




\subsection{Quantitative Demographic Data}
 


%%%%%%%%%%%%%%%%%%%%%%%%%%%%%%%%%%%%%%%%%%%%%%%%%%%%%%%% 
