
%%%%%%%%%%%%%%%%%%%%%%%%%
%   NSF RTG  2021
%  
%
%   Due:   June 1, 2021
%%%%%%%%%%%%%%%%%%%%%%%%%
\documentclass[11pt]{article}

\oddsidemargin 0.10in \evensidemargin -0.65in
\textwidth 6.2in         % Width of text line.
\topmargin 0.60in \headheight 0.0in \headsep 0.0in
\textheight 8.5in        % Height of text (including footnotes and figures,
\topskip 0.0in

%  \usepackage{showkeys}
\usepackage{color}
\usepackage{amsmath, amssymb, latexsym, natbib}
\usepackage{psfrag,epsfig,amsfonts,amsmath,latexsym,amsthm,amssymb,amscd,url }
\usepackage{amsmath}
\usepackage{bm}
 

\newcommand{\F}{{\mathcal{F}}}
\newcommand{\B}{{\mathcal{B}}}

\newcommand{\eps}{\varepsilon} 

\renewcommand{\k}{\kappa}
\newcommand{\p}{\partial}
\newcommand{\D}{\Delta}
\newcommand{\om}{\omega}
\newcommand{\Om}{\Omega}
\renewcommand{\phi}{\varphi}
\newcommand{\e}{\epsilon}
\renewcommand{\a}{\alpha}
\renewcommand{\b}{\beta}
\newcommand{\N}{{\mathbb N}}
\newcommand{\R}{{\mathbb R}}
\newcommand{\T}{{\mathbb T}}

\newcommand{\Le}{L_t^{\alpha}}

\newcommand{\EX}{\mathbb{E}}
\newcommand{\PX}{\mathbb{P}}


\newcommand{\grad}{\nabla}
\newcommand{\n}{\nabla}
\newcommand{\curl}{\nabla \times}
\newcommand{\dive}{\nabla \cdot}

\newcommand{\ddt}{\frac{d}{dt}}
\newcommand{\la}{{\lambda}}

\newcommand{\bu}{\mathbf{u}}

\newcommand{\obu}{\bar{\mathbf{u}}}
\newcommand{\bsigma}{\mathbf{\sigma}}
\newcommand{\btau}{\mathbf{\tau}}


\newcommand{\nd}{{\nabla \cdot}}
\newcommand{\dd}{\mathrm{d}}
\newcommand{\x}{{\bm x}}

\newcommand{\cF}{{\cal F}}
\newcommand{\cG}{{\cal G}}
\newcommand{\cD}{{\cal D}}
\newcommand{\cO}{{\cal O}}

%%%%%%%%%%%%%%

\newtheorem{theorem}{Theorem}
\newtheorem{lemma}{Lemma}
\newtheorem{definition}{Definition}
 \newtheorem{coro}[lemma]{Corollary}
 \newtheorem{example}[lemma]{Example}
 \newtheorem{remark}[lemma]{Remark}


%%%%%%%%%%%%%%% %%%%%%%%%%%%%



\newcommand{\uk}[1]{\ensuremath{u^{(#1)}(t,\omega)}}
\newcommand{\hse}{\ensuremath{h^s(\xi,\omega)}}
\newcommand{\hsk}[1]{\ensuremath{h^{(#1)}(\xi,\omega)}}

\newcommand{\sz}{\ensuremath{ {\int_s^0 z(\theta_r (\omega))\,dr}}}
\newcommand{\sZ}{\ensuremath{ {\int_s^0 Z(\theta_r (\omega))\,dr}}}
\newcommand{\zz}[1]{\ensuremath{{z(\theta_{#1} (\omega))}}}
\newcommand{\ZZ}[1]{\ensuremath{{Z(\theta_{#1} (\omega))}}}
\newcommand{\fu}[1]{\ensuremath{{F_u^{u_0 (#1)}}}}
\newcommand{\fus}[2]{\ensuremath{{\int^0_{#2} F_u^{u_0 (#1)}\,d{#1}}}}
\newcommand{\fuss}[2]{\ensuremath{{\int^{#2}_0 F_u^{u_0 (#1)}\,d{#1}}}}

\newcommand{\fuu}[1]{\ensuremath{{F_{uu}^{u_0(#1)}}}}
\newcommand{\rb}{\right)}
\newcommand{\lb}{\left(}
\newcommand{\rB}{\right]}
\newcommand{\lB}{\left[}


\newcommand{\nb}{\mathbf{n}}
\newcommand{\ub}{\mathbf{u}}
\newcommand{\xb}{\mathbf{x}}
\newcommand{\xnb}{\mathbf{x}_0}
\newcommand{\GaB}{\mathbf{\Gamma}}

\newcommand{\bo}{\mathcal {O}}
\newcommand{\so}{\mathcal {o}}

\newcommand{\BE}{\begin{equation}}
\newcommand{\EE}{\end{equation}}
\newcommand{\BEN}{\begin{equation*}}
\newcommand{\EEN}{\end{equation*}}
\newcommand{\BAL}{\begin{align}}
\newcommand{\EAL}{\end{align}}
\newcommand{\BAN}{\begin{align*}}
\newcommand{\EAN}

\newcommand{\s}{{\sigma}}
\def\Tr{\mbox{Tr}}
\newcommand{\Rn}{\mathbb{R}^n}

\DeclareMathOperator*{\argmax}{arg\,max}
\DeclareMathOperator*{\argmin}{arg\,min}
\newcommand{\FredNote}[1]{{\color{blue} Fred: #1}}

\iffalse
The long-range goal of the Research Training Groups in the Mathematical Sciences (RTG) program is to strengthen the nation's scientific competitiveness by increasing the number of well-prepared U.S. citizens, nationals, and permanent residents who pursue careers in the mathematical sciences, be they in academia, government, or industry. A significant part of this goal is to directly increase the proportion and the absolute number of U.S. students at the RTG sites who pursue graduate studies and complete advanced degrees in the mathematical sciences. It is anticipated that RTG projects also will serve as national models for research training in the mathematical sciences. Activities with potential impact beyond the directly-supported students and beyond the institutions receiving RTG funds will be key strengths in proposals. Collaborative proposals involving different types of programs (for example, institutions in which the relevant department does not award Ph.D.s, minority-serving institutions, etc.) and having the potential to develop innovative approaches to research training in the mathematical sciences are welcome. For such collaborative efforts, the lead institution must grant a doctoral degree in mathematical sciences.

The RTG program supports efforts to improve research training by involving undergraduate students, graduate students, postdoctoral associates, and faculty members in structured research groups anchored in a coherent research program. The activities need not be focused on a particular research problem; rather, it is expected that group participants will be united by common topical interests. The groups may include researchers and students from different departments and institutions, but the research-based training and education activities must be based in the mathematical sciences. RTG projects are expected to vary in size, scope, and proposed activities, as well as in their plans for organization, participation, and operation. However, research groups supported by RTG will include vertically-integrated activities that span the entire spectrum of educational levels from undergraduates through postdoctoral associates.
Addressing all stages (from undergraduate through postdoctoral) of trainee involvement is essential in RTG proposals. Proposals that focus on only one stage are not appropriate for submission to the RTG activity. While emphasis on graduate training in RTG projects is appropriate and natural, a substantial plan for involving undergraduates is necessary. When used in reference to undergraduates, the word "research" should be given its broadest interpretation.

Successful proposals will include collaborating faculty with a history of research accomplishments. This group should have a history of working with students and/or postdoctoral associates, and they should present a strong plan for recruiting students who are U.S. citizens, nationals, or permanent residents into their program. The RTG program is not meant to establish new research groups, but to enhance the training activities of existing groups with strong research records.

Graduate Traineeships. Graduate trainees form a pivotal component of the integration of activities in RTG grants. Their participation should result in:
1. involvement with research activities that include undergraduates, other graduate students, postdoctoral associates, and/or faculty members; 
2. graduate education that is both broad and deep; and
3. significant teaching or other professional experience such as industry/laboratory internship.
Mentoring, that is, guidance in professional development, is a critical strategy for preparing graduate trainees to become successful researchers, communicators, and mentors. Graduate trainees are expected to have substantial mentored professional experiences to prepare them for successful careers in the mathematical sciences and in other professions in which expertise in the mathematical sciences plays an important role. Examples of this professional experience could include:
• a minimum of two terms of supervised teaching, preferably with one term of more independent teaching in which the student has substantial responsibility for a class, or
• a minimum of two terms of a supervised industry/laboratory internship.
Some element of their activities should help students develop proficiency in the presentation of mathematical sciences research in both written and oral formats and in the ability to place their research in context.

RTG awards are intended to allow graduate students significant time for research, course work, and related activities. A graduate trainee can receive up to 33 months of non-teaching support from an RTG activity. RTG stipends cannot be used to pay students to fulfill teaching duties or for internships. Departments must demonstrate how the traineeships will improve the quality of the education their graduate students receive. The traineeships are not intended to replace existing institutional funding of research fellowships or scholarships.

Undergraduate Experience. In this program solicitation, the term "research experiences" for undergraduates includes all activities that involve undergraduates in discovery and generate appreciation of and excitement about research in the mathematical sciences. An undergraduate research experience does not have to result in the publication of a paper. Examples of research experiences include faculty-directed projects, either during the academic year or the summer, or participation in research teams with graduate students and/or postdoctoral associates. Such experiences are intended to involve students in the creative aspects of mathematical sciences in a non-classroom setting. They are also expected to enhance the development of students' communication skills, with particular emphasis on the presentation of mathematical concepts in both written and oral formats. In all cases, it is expected that the participating undergraduates receive mentoring to stimulate their further interest in the mathematical sciences.

Postdoctoral Associates. Effective RTG activities better prepare postdoctoral associates for their future careers. It is expected that at the end of the postdoctoral experience, each associate will have a well-defined independent research program, well-developed communication skills, a broad perspective of his or her field, and the ability to mentor.

The postdoctoral program can provide opportunities not traditionally found in mathematical sciences education and training, including interdisciplinary research experiences in connection with other departments and programs; participation in international research programs; internships in business, industry, or government laboratories; or participation in research institute programs suitably aligned with the associate's research interests. Postdoctoral associates are expected to teach, on average, one course per term while in residence at the sponsoring university. Over the duration of the postdoctoral appointment, this teaching should encompass a diverse set of instructional experiences at different levels of the curriculum. Likewise, it is expected that each RTG postdoctoral associate will submit a research proposal to a funding agency at some time during the course of the postdoctoral appointment. Mentoring to help ensure all postdoctoral associates become successful researchers, communicators, and mentors is a critical element of an RTG postdoctoral program, as is interaction of postdoctoral associates with undergraduate and/or graduate students.

The typical RTG postdoctoral appointment is for three years. A person is eligible for only one RTG postdoctoral appointment. An RTG postdoctoral associate is expected to be a recent recipient of a doctoral degree, typically held not more than three years as of January 1 of the year in which the appointment begins. Any exceptions made to this restriction should be well-justified in the annual reports.

Budget. Proposals may include support requests for graduate and advanced undergraduate students, postdoctoral associates, visitors, consultant services, travel, conferences, and workshops. Other budget items that are deemed to be essential to the success of the proposed activities may be included. Faculty salary is limited to that needed for the purpose of organizing and managing the program.

Data. RTG proposals will be strengthened by supporting data about the department's programs. An extensive discussion of the requested data appears in the Supplementary Documentation section below (V.A.7).

A successful RTG proposal will:
• be based in a U.S. IHE that grants the Ph.D. in the mathematical sciences (faculty and trainees from other types of institutions may be included through a collaborative proposal or other mechanisms);
• be anchored in a coherent research program in the mathematical sciences;
• have a realistic plan showing how the proposed activity would create new or enhanced research-based training experiences in the mathematical sciences for the students and postdoctoral associates;
• be directed by a principal investigator, with at least two other faculty members, who will collaborate in management and participate fully in the RTG activities.
A successful RTG proposal must convince reviewers that the project:
• integrates research with educational activities;
• provides for developing professional and personal skills, such as communication, teamwork, teaching, mentoring, and leadership;
• includes an administrative plan and organizational structure that ensures effective management of the project resources;
• has an institutional commitment to furthering the plans and goals of the RTG project and to create a supportive environment for integrative research and education;
• has a plan for recruitment, selection, and retention of participants, including members of underrepresented groups, so as to increase the number and diversity of U.S. citizens, nationals, and permanent residents in the graduate and postdoctoral programs;
• serves as a national model by effectively disseminating best practices for attraction, retention, and high-quality preparation of students and postdoctoral associates in the mathematical sciences; and
• has a post-RTG plan. The RTG program is intended to help stimulate and implement permanent positive changes in research training within the mathematical sciences in the U.S. Thus it is critical that an RTG site adequately plan how to continue the pursuit of RTG goals when funding terminates.
\fi

%%%%%%%%%%%%%%%%%%%%%%%%%%%%%%%%%%%%%%%%%%%%%%%%%%%%%%%%%%%%%%%%%%%%%%%%%%%%%%%%%%%%%%%%%%%%%%%%%%
\begin{document}  
\title{RTG: Dynamics -- Deterministic to Stochastic, Particles to Continua, Classical to Quantum }
\author{    }
%  Department of Applied Mathematics,
%\bigwedge   Illinois Institute of Technology     }
%Chicago, IL 60616 \\
%duan@iit.edu \\
% www.iit.edu/\textasciitilde duan}

\date{\today}
 
\maketitle

%\tableofcontents

%\newpage
\section{Introduction}

Although the basic undergraduate and graduate level mathematics curriculum provides sophisticated quantitative tools, it generally does not develop in students the \emph{mathematical maturity} to integrate these tools into a unified \emph{creative toolbox}. Developing this maturity is the overarching goal of this Research Training Grant (RTG).  

We will accomplish our goal by providing training that supports the strong, underlying linking theme of \emph{complex dynamical systems}.  These dynamical systems may be under random influences \cite{Arnold, Duan2015}, far from equilibrium \cite{liu2009introduction} and/or operate in quantum regimes \cite{Mbius1996DittrichWR}. Our approaches for understanding them may be particle (i.e., trajectory and sample path) based or continua (i.e., probability density, ensemble, energy, action functional, and wave) based.  Our training in state-of-the-art mathematical and computational tools will introduce to a spectrum of applications that is fundamental to current scientific and technological progress, such as geophysical and climate systems, liquid crystals, multi-phase flows, ionic fluids,  transportation systems, sensor networks, and biochemical reaction systems. 


This RTG builds on the research expertise among the PIs, and our existing applied mathematics curricula at Illinois Tech.  The research portfolio of our department covers a broad range of applied mathematics and statistics, and includes topics germane to this RTG, including dynamical systems, stochastic dynamical systems, multiscale and variational methods,  Monte Carlo methods, foundation of computational mathematics, and data-driven predictive modeling and simulation. Our work has applications in the physical, chemical, geophysical, biophysical, biological, and materials sciences, as well as finance.  Given that the four of the PIs are among the leadership of our department, we expect to the bulk of our department to participate in this RTG in some capacity.

% Our three research themes under the umbrella of complex dynamical systems are (i) dynamics under uncertainty; (ii) energetic variational approaches for complex dynamics; and (iii) mathematical tools for quantum dynamics. 

We currently have 65 undergraduate applied math majors and 21 PhD students. Our graduates build their careers in  both industry   and academia. Many of our undergraduates continue for PhD studies in other top universities such as Cornell, NYU, UCLA and UIUC. Many of our postdocs or visiting assistant professors get tenure track positions after leaving Illinois Tech.

This RTG will benefit from our existing collaboration with  Argonne National Laboratory (18 miles west of our campus) in joint research   efforts, seminars, and summer internships for our undergraduate and graduate students. Furthermore, the trainees in this RTG will have opportunities to participate in relevant academic, educational or outreach programs at the NSF Institute for Mathematical and Statistical Innovation (IMSI), hosted at University of Chicago (5 miles south of our campus).

% Talk about this later:  In fact, IMSI has encouraged us to propose a joint  workshop relevant to our RTG theme,   thus helping publicize and recruit students and postdocs for this RTG.

%Other entities at Illinois Tech relevant to this RTG include the  Center for Interdisciplinary Scientific Computation,   Laboratory for Stochastic Dynamics and Computation, Center for Learning Innovation, and Career Services. The Camras Scholars Program help recruits and retain talented undergraduate students. Our Interprofessional Projects (IPRO) Program provides the alternative to a traditional undergraduate education. Our signature IPRO Program remains one of just a few programs of its kind in the country. IPRO joins students from various majors to work together to solve real-world problems, often on behalf of sponsor companies and nonprofits. A required academic program, IPRO teaches leadership, creativity, teamwork, design thinking, and project management—uniquely preparing students to succeed in a professional work environment. Moreover, the  Pritzker Institute of Biomedical Science and Engineering provides a limited number of research stipends for undergraduate math majors to conduct research in biophysical modeling. 
   
%All of our 18 tenure-stream faculty are actively engaged in research and teaching. 

%All faculty are fully engaged in teaching both undergraduate and graduate studies.   Current faculty members (Names?) have received the highest teaching recognition:.... Distinguished Teaching Professor Award. The PI Duan once received a Teaching Innovation Award. We also have four postdocs or visiting assistant professors with significant teaching responsibilities. 

%With this in mind, it is not unrealistic to expect that in one form or another, all Math faculty will participate in this RTG program, either as research team co-leaders, participants, or teaching mentors. In the next section and in the budget, we will identify those colleagues who will carry out specific RTG related tasks.

%We are in an ideal location    for research  training activities. For example, we hosted 3 CBMS-NSF research conferences and attracted .... 




  

\iffalse 
\subsection*{Illinois Tech resources  for this RTG}
Our university has the following resources to support this RTG.

 $\bullet$ Center for Interdisciplinary Scientific Computation
 
 $\bullet$ Laboratory for Stochastic Dynamics and Computation
 
 $\bullet$ Pritzker Institute of Biomedical Science and Engineering
 
  $\bullet$  The Camras Scholars Program
  
  This Program help recruits and retain talented undergraduate students. 
  
  $\bullet$  Center for Learning Innovation 
 
 
 $\bullet$   Career Services
 \fi
 
% Argonne National Laboratory resources ....
 
 




%%%%%%%%%%%%%%%%%%%%%%%%%%%%%%%%%%%%%%%%%%%%%%%%%%%%%%%%
\section{Proposed Project  }
    


We propose three research projects for this RTG: (i) Dynamics  under Uncertainty, (ii)  Energetic Variational Approaches for Complex Systems,  and (iii) Mathematical Tools for Quantum Dynamics.
 
  

 The team for each project will consist of research faculty with at least one member
from Argonne National Lab or other departments, Illinois Tech math undergraduate and graduate students, and a postdoctoral
fellow. Whenever possible, other undergraduate and high-school participants of summer activities
may be part of year-long activities. While there  
 may be thematic driven differences on the approach to deliver
knowledge and do research, each group share a similar year-round schedule that will allow awareness
and exchange of ideas throughout the academic year, mainly at the RTG graduate seminar. There will also
be a set of common courses that foment shared knowledge. Finally each summer program is centered on
an intense 2-week period. Participating undergraduates will have a unique opportunity to work in a serious
way on two topics, which we believe to be highly beneficial.


\subsection{Research Projects}


\subsubsection*{Project 1: Dynamics  under Uncertainty} 

{\bf Team:} Jinqiao Duan, Fred Hickernell, Michael Pelsmajer, Romit Maulik (Illinois Tech and Argonne National Lab),  
 Mustafa Bilgic (Illinois Tech-CS) and Prasanna Balaprakash (Argonne National Lab), together with one or more undergraduate students, one or more PhD students, and a postdoctoral fellow.  

{\bf Objective:} The interactions of uncertainty and nonlinearity lead to intriguing phenomena, such as  transitions   between  different dynamical regimes. We develop modeling, analytical, computational and data science  methods to quantify     stochastic dynamics, with applications to chemical \cite{agaoglou_chemical_2019}, biophysical \cite{Ruoff2018BiologicalCR}, powergrid infrastructures \cite{MEDJROUBI201714}, and climate systems \cite{Alexandrov2020NonlinearCD, franzke_o'kane_2017, Wan2020ADF}. 
 
 
 

 
 Complex systems are oftentimes under the influence  of randomness or uncertainty \cite{Moss1, Horst, Gar, VanKampen3}. Uncertainties may also be caused by our lack of knowledge of some physical processes that are not well represented in the mathematical models (epistemic) and due to the inherent randomness of a specific event  \cite{Palmer1, Kantz, Wilks, Williams}.
Although these random mechanisms appear to be very small or very fast, their long time impact on the system evolution may be delicate or even profound \cite{Arnold, DuanBook2015}. These delicate impacts on the overall evolution of dynamical systems has been observed in, for example, stochastic bifurcation
\cite{Crauel, CarLanRob01, Horst}, stochastic resonance \cite{imkeller2002model},
 and  noise-induced pattern formation \cite{Gar, blomker2003pattern}.
Hence taking stochastic effects   into account is of
central importance for mathematical modeling of
complex systems under uncertainty, and this leads to stochastic   differential equations  \cite{Arnold, Ikeda, Okse2003, WaymireDuan}. We often think these are the governing equations for stochastic `particles'. It is therefore crucial to investigate dynamics under uncertainty, in the context of models arising from applications in, for example, geophysical and climate systems.
Fluctuations in complex systems are often
{\bf non-Gaussian} \cite{Woy,Dit,Swinney,Shlesinger,taqqu,dybiec2009levy} rather than Gaussian (modeled by Brownian motion). Non-Gaussian fluctuations are often modeled by $\alpha$-L\'evy motions (for $0<\alpha<2$). This leads to 
 stochastic   differential equations  \cite{Arnold, Ikeda, Okse2003, WaymireDuan}
 with L\'evy motion as well as Brownian motion. We often think these   the governing equations for stochastic `particles'.

%For instance, it has been argued that diffusion by geophysical turbulence \cite{Shlesinger} corresponds  to a series of  ``pauses", when the particle is trapped by a coherent structure, and ``flights" or ``jumps" or other extreme events, when the particle moves in the jet flow. Paleoclimatic data \cite{Dit} also indicate such irregular processes. There are also experimental demonstrations of L\'evy flights  in optimal foraging theory and rapid geographical spread of emergent infectious disease.   Humphries {\it et. al.} \cite{Humphries}   used GPS to track the wandering black bowed albatrosses around an Island in Southern Indian Ocean to study the movement patterns of searching food.   They found that by fitting the data of the movement steps, the movement patterns obeys the power-law property with power parameter $\alpha=1.25$. To get the data set of human mobility that covers all length scales,  Brockmann   \cite{Brockmann}  collected data by online bill trackers, which   give successive spatial-temporal trajectory with a very high resolution. When fitting the data of probability of bill traveling at certain distances within a short period of time (less than one week), they found power-law distribution property with power parameter $\alpha=1.6$.

%This parameter is of great importance since by using it into the classic SIS model, they found probability density function patterns generated by non-local dispersal:  $\alpha-$stable L\'evy motions are strikingly similar to practical data of human influenza. A further example is a thermally activated motion of a test particle along a polymer,   shown in \cite{sokolov1997paradoxal, brockmann2002levy} to be subject to $\frac{1}{2}$-stable L\'evy motion due to polymer's self-intersections.  This motivates the investigation of dynamical systems driven by both Gaussian and non-Gaussian   fluctuations, especially the heavy-tailed,  $\alpha-$stable   L\'evy motions.  

 
 \textbf{Quantifying transition phenomena}:
To understand transition   in    a stochastic system, we  
  study the     'ensemble' (or `continua') quantities: Mean exit time (exiting from a dynamical regime),   escape probability (from one dynamical regime to another),    and  the transition  probability density.
This leads to deterministic  nonlocal (integro-differential) equations for mean exit time and for escape probability, and a nonlocal Fokker-Planck equation,
in which the usual Laplace operator is replaced
by a nonlocal (or fractional) Laplace operator:  $-(-\Delta)^{\frac{\alpha}2  }$, for $\alpha \in (0, 2)$.
 
%Both mean exit time and escape probability are described by nonlocal `elliptic' partial differential equations, with unusual boundary conditions corresponding to absorbing or escaping situations at microscopic levels. The probability density function for the solution of an SDE with  with L\'evy noise is described by a nonlocal Fokker-Planck equation.
  
As the usual numerical methods (such as finite difference) are non feasible, we develop a method to simulate these high dimensional nonlocal equations based on a stationary or temporal normalizing flow model which is a subset of generative machine learning models using deep learning. This method involves techniques to handle the nonlocal terms implicitly since arbitrary densities can be learned from samples alone. Subsequently, `ensemble' quantities, including the mean trajectories (when exist), are high dimensional integrals with respect to the probability density \cite{DuanBook2015}.  Thus we validate our fit distributions with our Monte Carlo simulation results \cite{MCQMC2020QMCPyTut, QMCPyTutColab2020}, taking advantage of recent advances in mathematical foundation of high dimensional numerical quadrature \cite{HicEtal17a}. 

%Machine learning for high dimensional stochastic dynamics using deterministic quantities or indexes which are defined as mean or high dimensional integrals.

 
\textbf{Extracting transition phenomena from noisy data}:
Some   stochastic systems may be only known or partially known (where some mechanisms are missing) by   noisy observations or simulated data sets. We will  develop techniques to extract stochastic governing laws \cite{YangLi2020a} and then examine transition phenomena by computing mean exit time, escape probability and transition probability density as above. 
There are several potential interfaces with state-of-the-art generative machine learning here which has been constructing novel methods to learn densities from data as neural stochastic differential equations. Furthermore, our colleagues'  works on 
designing and optimizing sensor networks, and graph theory \cite{karwa2016statistical,Calines2008MonitoringSF} will be explored to guide smart data collection for interface with such novel algorithms. 

%between distinguished states \cite{DaiMinChaos,LuYB2020, ZhengDuan2017, HuangYF, ZhengYY2020}.  and the Onsager-Machlup action functionals \cite{ChaoDuanOM, HuangYF2020}.  We will combine our group member' research in networks and    graph theory,  to reveal dynamical transitions between distinguished states (e.g., metastable states) of these large networks.

A class of high dimensional dynamic systems may be interpreted as networks, such as large gene regulation networks \cite{Raser2005} which are modeled by systems of stochastic differential equations \cite{Suel06}.  In a gene regulation network, a transition from low to high concentration of a promoter protein may indicate transcription \cite{Stefan,ZLDK}. Studying transitions in large networks is also of vital importance for resiliency analyses and control of powergrid infrastructure and for supply chain modeling in operations research which may be modeled as stochastic processes on graphs \cite{shin2020graph,anghel2007stochastic,nardelli2014models}. For large networks with multiple sources, sinks and connectivity measures of several variables, extremely high dimensional systems are commonly obtained. Furthermore, optimization of these graphs for desired quantities of interest require the solution of a complex mixed-integer nonlinear programming problem with constraints that may not be nominal input-output mappings but aforementioned complex dynamical systems \cite{shin2020decentralized,sampat2017optimization,kim2019graph,shin2021exponential}.

\textbf{A new course on High Dimensional Stochastic Dynamics via Machine Learning:}
A new graduate course will be developed on modeling of transport processes
in high dimensional stochastic systems using data-driven algorithms.  The course will fill a gap by addressing simulation methods for nonlocal equations in high dimensional domain and validate with a Monte Carlo  method and thus better prepare the students for future careers in both industry and academia.
 
 
 \textbf{Regular group activities:} We propose to establish a bi-weekly  
journal club. The key objectives will be to have students present articles relevant for the main topics of the
project followed by a discussion of how the methods developed in our research could be applied to emerging
areas of   research (e.g., gene networks, climate models). This will enhance our students’ grasp of the important current research topics
and potentially lead to new collaborations. We also propose that graduate students and postdoc regularly
make short presentations on their research progress to all RTG participants, followed by a discussion of next steps in each individual research project.


 \textbf{Summer program:} The team will organize two-week summer programs for undergraduate students with
the focus on stochastic mathematical models used to describe transport processes.
The first week will start with two lectures by RTG math faculty. These will provide the general background
information in the form accessible to undergraduate students. We envision one lecture to focus on developing
models for physical systems and the other lecture to be primarily addressing the basic concepts behind the
numerical methods used for simulations of such systems. For the rest of the two weeks, the focus will be on
three sets of activities: (i) seminar-style presentations prepared by graduate students and postdoc, describing
specific topics related to the ongoing research projects; (ii) undergraduate students working in teams on small
research projects chosen in the beginning in the first week under the guidance of RTG faculty; (iii) visiting
  Argonne National Laboratory and Pritzker Institute for Biomedical Science and Engineering and interacting with the faculty members, students, and
postdocs conducting experimental works, with the objective of getting better understanding of the physical
systems being modeled in the framework of the project. Each team of the undergraduate students will make
a short presentation of the results of their work in the end of the second week.


 \textbf{Yearly workshop:} A two-day workshop will be organized in the end of each academic year. Each workshop will consist   of presentations by graduate
students and postdocs, with one or two invited lectures (funded internally by the math department) given
by top researchers. These researchers will be chosen from experts.   





\subsubsection*{Project 2: Energetic Variational Approaches for Complex Dynamics}
\noindent
{\bf Team.} Chun Liu, Yiwei Wang (Illinois Tech, Applied Math), together with 1 postdoc, 2 graduate students and 1 undergraduate student.

\medskip


\noindent {\bf Objective:} The research focus will be on the applications of the energetic variational approach in mathematical modeling, nonlinear partial differential equation, scientific computing and machine learning. The framework of energetic variational approach, originated from seminal works of Raleigh \cite{strutt1871some} and Onsager \cite{onsager1931reciprocal,onsager1931reciprocal2}, provides a paradigm to determine the dynamics of a non-equilibrium system from a prescribed energy-dissipation laws.%non-equilibrium systems for given 
 The variational principle has been employed to study many complex fluids, such as liquid crystals, two-phase flows, and ionic fluids, which  involve the coupling and competition of various mechanical and chemical mechanisms in different scales \cite{lin2001static, feng2005energetic, LiLiZh05, Lin2007, liu2009introduction, du2009energetic, sun2009energetic, eisenberg2010energy, Giga2017, Liu2019, knopf2020phase}.

For an isothermal closed system, an energy-dissipation law, comes from the first and second law of thermodynamics, is often given by $\frac{\dd}{\dd t} E^{\rm total} = - \triangle$,
where $E^{\rm total}$ is the total energy, including both the kinetic energy $\mathcal{K}$ and the Helmholtz free energy $\mathcal{F}$, and $\triangle \geq 0$ is the rate of the energy dissipation which is equal to the entropy production in the situation. The energy-dissipation law, along with the kinematics relation, %describe all the physics and the assumptions for a given non-equilibrium system. 
capture all specific physical and biological properties for a given non-equilibrium system. Starting with an energy-dissipation law, the EnVarA framework derives the dynamics of the systems through two variational principles, the Least Action Principle (LAP) and the Maximum Dissipation Principle (MDP). The LAP, which states the equation of motion for a Hamiltonian system
can be derived from the variation of the action functional $\mathcal{A} = \int_{0}^T \mathcal{K} - \mathcal{F} \dd t$ with respect to the flow maps, gives a unique procedure to derive the conservative force for the system. The LAP is indeed an manifestation of the rule $\delta E = {\rm force} \cdot \delta x$. The MDP, variation of the dissipation potential $\mathcal{D}$, which equals to $\frac{1}{2}\triangle$ in the linear response regime, with respect to the rate (such as velocity), gives the dissipation force for the system. In turn, the force balance condition leads to the evolution equation to the system
\begin{equation*}
\frac{\delta \mathcal{D}}{\delta \x_t} = \frac{\delta \mathcal{A}}{\delta \x}.
\end{equation*}
% An advantage of using EnVarA the coupling and competition of  in different scales.
The EnVarA can systematically deal with coupling and competition of mechanical, chemical and thermal mechanisms in different time scale in a thermodynamically consistent framework. Moreover, it provides an framework to quantify energy transduction between different processes.  

\emph{The RTG faculty has the experience and relevant expertise to lead the project, as evidenced by relevant publications. We plan to focus on the modeling of active soft materials} Thermal effects ..

\emph{The interdisciplinary nature of the team will provide an opportunity to have balanced training in different approaches of applied math, most notably multiscale modeling and analysis arising for the problems in biology and soft matter physics.}



\noindent {\bf Introduction of Mathematical Theories of Complex Fluids  Course:}  A new graduate course will be developed on mathematical theories of complex fluids. Complex fluids is ubiquitous in our daily life and important in many industrial, physical and biological applications. Studying these materials requires a wide range 
of tools and techniques for different disciplinary, even within mathematics. The course will covers Basic mechanics, non-equilibrium thermodynamics, multi-scale modeling and analysis, and modeling for liquid crystals, ionic fluids. The course will fill the gaps between ... and ... 



\iffalse
1. General energetic variational framework for complex fluids: 
      a) least action principle and maximum dissipation principle. 
      b) Navier-Stokes equations and elasticity. 
      c) viscoelastic materials: nonlinear elasticity, incompressible elasticity, viscoelasticity. 
      d) generalized diffusion, nonlocal diffusion.

2. Free interface motion in the mixture of different fluids: 
     a) conventional description: sharp interface description, water wave, vortex sheet, surface tensions. 
     b) diffusive interface description: microscopic background (self-consistent field theory), Flory-Higgins theory, 
         sharp interface limit, dynamics. 
     c) slippery boundary conditions and systems on (or near) surfaces. 
     d) Helfrich elastic bending energy and application to vesicle membranes. 

3. Multiscale modeling and analysis: 
     a) basis of stochastic differential equations: Fokker-Planck equations, diffusion, 
         Smoluchowski coagulation equations, variational formulations, kinetic theory. 
     b) micro-macro models for polymeric materials. 
     c) moment closure methods, Mori-Zwanzig formulation and other coarse grain methods. 

4. Ionic fluids and ion channels: 
     a) electroeheological (ER) fluids: Poisson-Boltzman fluids, like charge attaction (LCA) and charge inversion, 
         steric effects of ion particles, density function theory of Rosenfeld, equation of states. 
     b) basic physiology of ion channels and protein structures. 
     c) ionic fluids in ion channels. 
     d) ionic osmosis in biological systems. 
     \fi
     
\noindent {\bf Summer REU}: Liu is coordinating a summer REU program for underrepresented students in Chicago.
He collaborated with local community colleges and Northeastern Illinois University for recruiting
and training of these students.

\noindent {\bf Regular group activities}: \emph{We propose to establish a weekly Multiscale Seminar. The key objectives will be to have students present articles relevant for the main topics of the project followed by a discussion of how the methods developed in our research could be applied to emerging areas of experimental research. This will enhance our students’ grasp of the important current research topics and potentially lead to new collaborations. We also propose that graduate students and postdoc regularly make short presentations on their research progress to all RTG participants, followed by a discussion of next steps in each individual research project.}

\noindent {\bf Yearly workshop}:  A two-day workshop will be organized....




\subsubsection*{Project 3: Mathematical Tools for Quantum Dynamics}
\noindent
{\bf Team.}  Jinqiao Duan, Fred Hickernell, Romit Maulik (Illinois Tech and Argonne National Lab), Carlo Segre (Illinois Tech-Phys), and Martin Suchara (Argonne National Lab).
\medskip

\noindent
{\bf Objective.} The development of quantum science and its exploitation in technology is a topic of great interest and activity in many countries. The United States \cite{raymer2019us}, Europe \cite{riedel2019europe}, and China \cite{kania2018quantum} have invested heavily in long term ``quantum initiatives’’.  It has been stated that we are in the midst of the ``second quantum revolution’’ \cite{kania2018quantum}. Indeed,  most people tend not to realize just how much quantum mechanics influences their day-to-day life. For example, consider that standard household light bulb. Advances in light emitting diode (LED) technology have led to ``white’’ light bulbs that can now be purchased anywhere light bulbs are purchased. These LED light bulbs are more energy efficient, last longer, and are more environmentally friendly than light bulbs of the past. This is all thanks to quantum mechanics, which is fundamental to LED physics and technology. The impact of this rather mundane example is significant. But the impact of more complex physical and technological quantum advances, such as quantum computing, quantum cryptography, quantum sensors \cite{ng2020guest}, and even quantum artificial intelligence \cite{taylor2020quantum} will change the world and our way of life.

Therefore it is essential in education nowadays that students develop a level of ``quantum literacy’’ \cite{foti2021quantum} that will enable them to interpret and evaluate these new phenomena and technologies in a way that will enable them to assess their impact, not only on their own  lives, but on society and the world. Developing this quantum literacy in the context of mathematical education,    at both the undergraduate and graduate   levels,  is a goal of this RTG project.


Quantum mechanics is not typically emphasized in most mathematics programs.  Indeed, many of the unusual properties of quantum mechanics, such as tunneling, entanglement, superposition of multiple states, interference of states, uncertainty, and quantum measurement can be understood from the fundamental physical principle of ``wave-particle’’ duality and de Broglie’s relation between the momentum and the wavelength of a particle.  From this one is led to the Schr\”odinger equation description of particles in terms of waves. In fact, the physical setting of quantum mechanics is described by a few (mathematical) axioms that describe the state space, observable quantities, measurement, and time evolution.  Once this physical setting is translated into mathematics, the implications and consequences of the mathematics is a natural setting for the participation of mathematicians. However, the required mathematical skills and knowledge are not typically emphasized or even covered in the relevant courses. A goal of this RTG is to remedy this situation, both through training undergraduate, graduate, and postdoctoral students, and through the development of a course appropriate for mathematicians, but which bridges this gap. This course will lead naturally into a variety of research areas that fit naturally into the context of the other research projects, that we will describe.

%\smallskip
%\noindent
%{\bf Linear Algebra from the Quantum Point of View.} It cannot be emphasized enough that Linear Algebra (Linear Space Theory in general) is fundamental to many areas of pure, applied, and computational mathematics, and it is also fundamental to the structure of quantum mechanics. Of course, ‘’every’’ mathematics department has courses in Linear Algebra. However, topics important to quantum mechanics are often omitted.  Our course will fill that gap. The starting  point will be to emphasize that the basic structure of quantum mechanics is complex in nature—the solution of Schr\”{o}dinger’s equation is complex valued and linear spaces are complex vector spaces. The language that we will use is Dirac notation.  This complex structure has implications for the nature of inner products, and this plays a key role in the mathematical structure of quantum mechanics.  Multi-particle quantum systems are essential for the notion of entanglement. In classical mechanics multi-particle systems are described by the cartesian product of state spaces for each particle. In quantum mechanics they are described by a tensor product, and this algebraic structure tends not to be treated in traditional linear algebra courses, but it is essential for describing the basic elements, and showing the advantages of, quantum computation, and this mathematical structure will be emphasized in our course.  The Schmidt decomposition provides an insightful characterization of entanglement based on tensor product spaces and its proof relies on the singular value decomposition, another topic of great importance (also in data science) that is typically neglected in a first linear algebra course, but it will be emphasized in our course.


\medskip
\noindent
{\bf  Mathematical Framework for Quantum Mechanics.} This will be a course that is suitable for mathematics graduate students. We will build this course from  a   undergraduate course   taught by team member Carlo Segre, and another one taught by our Research Professor Stephen Wiggins (available on YouTube). PI Duan   taught a minicourse in Quantum Dynamics \cite{Gutzwiller1990ChaosIC, Holland1993TheQT,Micha2006QuantumDW} in 2019 and 2020 for graduate students and his collaborators.  We will invite our collaborators to co-teach this course. 
%The course is now available on YouTube \cite{https://www.youtube.com/playlist?list=PL6hB9Fh0Z1ELQy4WdFzlMcMD_XIb7C3no}, and has its own open source textbook and solutions manual available from FigShare. 
The YouTube format provides the opportunity to engage the students in an active learning format where they view the “short" lectures ahead of time and the in person class time  also includes  questions, discussions, and problem solving.


%\medskip
%\noindent
%{\bf The Phase Space Framework for Chemical Reaction Dynamics.} An extremely active and constantly evolving area that spans many disciplines of science of the study of the ‘’dynamics of transformation’’, of which chemical reaction dynamics theory is a central example. In recent years it has been shown that the phase space perspective of Hamiltonian dynamics can provide not only valuable insights but also predictive power. Moreover, this phase space approach provides a framework for understanding when quantum effects are important through the phase space formulation of quantum mechanics due Wigner, Weyl, Moyal, Groenewold, and others \cite{ waalkens2007wigner}. Material for this course is covered in our Jupyter book  \cite{agaoglou_chemical_2019}, which is freely available.   \cite{Beard2008ChemicalBQ}

\medskip
\noindent
{\bf Research Topics.} Below we list   research projects that can be  carried out in the context of the training we have offered.  We have experience tailoring such projects for the undergraduate, graduate, and postdoctoral levels.
 

\smallskip
\noindent{\bf Explorations of the Interplay Between Quantum Mechanical Uncertainty and Noise.} Uncertainty relations between quantum mechanical observables (such as position and momentum) arise naturally as a result of the mathematical formulation of quantum mechanics, in the absence of noise \cite{Griffiths2018IntroductionTQ}. This project will explore the interplay between quantum mechanical uncertainty and the influence of external noise \cite{Abend2016StochasticPI} on the quantum mechanical system. This may benefit from our recent work on effective quantum wave factorization \cite{ZHANG2020132573}. This also relates to the “Dynamics under Uncertainty’’ project.

%\smallskip
%\noindent{\bf Simple Models Describing the Phase Space Formulation of Quantum Mechanics.} The potential well problems in the first two projects are formulated and solved in the standard Schr\”{o}dinger equation setting.  This project will explore reformulating those problems in the phase space setting using the Wigner function approach. The two approaches will be compared and contrasted.

\smallskip
\noindent{\bf Applications of Methods of Lagrangian Transport in Fluids to the Pilot Wave Formulation of Quantum Mechanics.}   A `trajectory or particle' formulation of quantum mechanics  equivalent to the Schr\”{o}dinger equation can be given \cite{Bohmian, Holland1993TheQT} that emphasizes the flow of probability, or ``probability current’’ in space. This is the pilot-wave, or hydrodynamical, point of view. We will explore this point of view from the perspective of the dynamical systems approach to Lagrangian transport in fluid mechanics, including an examination of quantum dynamical transition phenomena \cite{waalkens2007wigner,Micha2006QuantumDW, Mbius1996DittrichWR}. 

%This is related to the ‘’Lagrangian transport’’ project.

%\smallskip

%\noindent
%{\bf Explorations of the Role of Bifurcations of Equilibria in Hamiltonian Systems on Classical Reaction Dynamics.} For two degree-of-freedom Hamiltonian systems the saddle-node, pitchfork, and Hamiltonian-Hopf  (or the (‘’Meyer-Schmidt’’) bifurcations typically occur.  We will explore how the normal forms for these bifurcations occur in chemical reaction dynamics and their influence on reaction.
  
%\item[Explorations of the Role of Bifurcations of Equilibria in Hamiltonian Systems on Quantum Reaction Dynamics.] This project will consider the classical normal forms of the previous project and the quantum mechanical manifestations of these bifurcations in reaction dynamics.

\smallskip

Our training and research in this project will use the open source software QuTiP \url{http://qutip.org} and Qiskit \url{https://qiskit.org}, which are now standard tools for modelling and simulation of quantum phenomena.



\subsection{Partnership with NSF Math Institute IMSI \& Argonne National Lab}

With letters of Collaboration from NSF Math Institute for Mathematical and Statistical Innovation (IMSI) and from Argonne, we will\\
\textbf{Organize a workshop on Mathematical Tools for Quantum Dynamics   at IMSI}, and \\
\textbf{Collaborate with Argonne} on machine learning methods for stochastic dynamics, and quantum mechanics in the context of applied mathematics. 




\subsection{RTG undergraduate training }
 
We offer a major in  Applied Mathematics with the possibility of a variety of specializations as well as options for double majors and co-terminal joint BS-MS degree. Starting from the fall of 2018, we will also offer a major in statistics. The undergraduate majors have numbered between 45 to 55 over the past five years. With this RTG, we goal is to increase this number and to get better prepared and diverse incoming students, establishing pathways of engagement between the department and the high schools around Chicago area through a variety of consistent outreach activities.



Our graduates build their careers in both industry and academia. We will continue to nurture this comprehensive training by providing opportunities to learn and work in either environment. Graduate students will be encouraged present their research in conferences. We aim to ensure that each student participates in academic or industry conferences before graduation. A significant effort 
will be devoted to creating a mentorship program that will help our graduate students to find their career path.

In the past five years, our RTG group faculty members have
supervised research for over a dozen undergraduate students, many of them got their research published and went on to Ph.D. studies at top universities such as    UCLA and UIUC.   One faculty member also supervised research for a high school student.  

%Track record for students supported by Pritzker Institute for Biomedical Engineering and Science, and IIT IPPO project courses, and ANL  (some went to NYU, Courant, ...) 

\subsubsection*{Undergraduate  academic year program  }
We will create a monthly RTG-Undergraduate evening event for our   math majors. In the early fall,
we will first give an overall presentation describing the RTG program with the goal of recruiting students.
Subsequent meetings will include presentations from the   Career  Services, the Camras Scholars
Program, and    Argonne National Laboratory's Undergraduate Program. By late fall, we will give brief presentations of the
research topics of the RTG program, which will serve as an introduction of faculty members of the different
groups. At the end of the fall term and for the first meeting in the spring, the meetings will concentrate on
training and helping students apply to REUs and graduate programs. Throughout the academic year, there
will be research presentations by RTG participants, mostly graduate students and postdoctoral fellows, with
more emphasis during the spring term. This last component serves two main purposes: to expose undergraduate
students to research and to prepare students for the four-week RTG summer program. The main
objective of this activity will be to increase the number of  math majors applying to PhD programs.
RTG funds will be used to provide food for these events, materials for presentations and advertising.

Our Interprofessional Projects (IPRO) Program provides the alternative to a traditional undergraduate education. Our signature IPRO Program remains one of just a few programs of its kind in the country. IPRO joins students from various majors to work together to solve real-world problems, often on behalf of sponsor companies and nonprofits. A required academic program, IPRO teaches leadership, creativity, teamwork, design thinking, and project management—uniquely preparing students to succeed in a professional work environment. Moreover, the  Pritzker Institute of Biomedical Science and Engineering provides a limited number of research stipends for undergraduate math majors to conduct research in biophysical modeling. 


\subsubsection*{RTG Summer Program}
With support from the RTG program we will run an in-house four-week undergraduate summer program.
Targeted students will be sophomores and juniors. Each year the program will run multiple sessions, each on a single RTG topic. They will run during the summers of Years 1 to 3 so that each topic
will be presented twice. Particulars on the format and content of each course are described in Section 2.1.
We will support up to ten students. We will budget for summer programs for Years 4 and 5 as well, but we reserve
the right to update content within the core RTG topics. For each session, there will be a faculty member
  overseeing the smooth running of the summer program. It is expected that most of the lectures
and supervision of small projects will be done by a graduate student and a postdoctoral fellow. RTG funds
will pay for room and board to participating students.

\medskip

 Summer Undergraduate Research Experience (SURE)

Sponsored by the College of Computing, the Department of Applied
Mathematics, together with Department of Computer Sciences, will run a
summer research experience program for undergraduate students.

SURE offers insight and learning in some of the hottest topics in
applied mathematics and computer science, including data science inspired by dynamics under uncertainty, variational modeling, and quantum mechanics,   through hands-on research
experiences. Learn to work as a team member by interacting with graduate
students and faculty, and gain an understanding as to what it takes to
conduct real-world research.

One of the goals of the program is to support first-generation,
low-income, and underrepresented students for STEM research and learning.
We are working with the neighboring community colleges and other
institutions in both recruiting and teaching of these students.
 


\subsection{RTG graduate training }

 
Our department has a single PhD program in Applied  Mathematics, focusing  
  on applied analysis,     computational mathematics,    statistics, and discrete mathematics. 

A strong and active PhD program is the backbone of a research-active mathematics department. PhD students act as force multipliers in research programs of their advisors as well as undergraduate instructional support of the department. The two key aspects of a successful such program are the appropriate number of students and comprehensive career-training of those students.
A strong and active PhD program is the backbone of a research-active mathematics department. PhD students act as force multipliers in research programs of their advisors as well as undergraduates.

We currently have 21 PhD students. With 18 faculty, our faculty-to-PhD-student ratio is close to 1-to-1.1, and far from a healthy 1-to-2 such ratio in most national research math departments. To increase our student strength to at least 35, we have to substantially increase our TA lines from the current 14.

Finally, our data show that the majority of our PhD recipients’ first job is in a non-academic position.
We view our success in placing students in a broad range of institutions as a unique and positive indicator,
which we want to maintain. The research topics and interdisciplinary approach should enhance opportunities
for jobs in industry and national laboratories for our RTG students after graduation, and at the same time
increase the number of PhD recipients going into high quality postdoctoral positions.


Each year, the RTG program will support three to four PhD students, each for their first three years in the
PhD program. RTG students will be released from typical duties that pay for traditional assistantships, so
that they can be fully committed to their coursework and regular RTG activities, primarily in the form of
seminars and participation in one working group. By the end of Year 1, each RTG student will enroll in one
of the groups and be assigned a faculty mentor from the corresponding team. They will be encouraged to
participate in the summer program and will receive one month support to work on a research project. Year 2
will still be focused on coursework with the expectation of continuing progress toward degree completion.
Ideally, students will do an internship during the summer of Year 2 or Year 3. Students who do not have
a summer internship will do in-house research along with active participation in the RTG summer program
in the form of giving lectures and mentoring undergraduate students. By the end of Year 3, RTG students
should pass qualifying exams and present their PhD proposal. They should also have participated in one to
two workshops run by the Center for Learning Innovation. In Year 4, students will be fully engaged in the dissertation work, expected
to be in one of the core programs. They will also teach a section of Calculus with the support of a teaching
mentor, who will observe a few classes and write a report. Summer 4 and Year 5 should focus on completing
their dissertation and applying for jobs and postdoctoral fellowships.

RTG graduate students will have at least two semesters of supervised teaching or internship at Argonne National Lab (or other government Labs and data science companies such as FactSet). 

 

\textbf{Mentoring/Advising:} Each RTG graduate student will meet once a semester with our Director of Graduate
Studies to monitor progress towards completion of the PhD. Selection of courses and committees of studies
will be decided in conjunction with an assigned RTG faculty mentor who will also be responsible in identifying
suitable summer internships.   As for all PhD students, a faculty member will be asked to observe their
teaching, provide feedback and prepare a report when applicable. RTG students will receive timely training
on topics such as application to postdoc fellowships and industrial jobs.
 

\textbf{RTG Seminars and Communication Skills}
We will organize RTG  weekly   seminar ``Frontiers in Applied Mathematics" for all RTG trainees. It
will feature talks, mini-courses, and guest lectures aimed at fostering collaboration between all RTG teams.
Consistent with the ongoing format, the responsibility of running it will be shared by RTG graduate students.
Once a year, the RTG-graduate team will organize a seminar under the umbrella of the Illinois Tech SIAM Student Chapter.  

RTG trainees  will improve oral and written     communication skills by presenting their own research works and learning from other presenters. Some presentations will be about how to apply for REUs, graduate schools, and sharing experiences for job search in academia and industry. 

\textbf{RTG Virtural Learning and Research: Zoom and YouTube}
We plan to hold  online Zoom research discussions and mini-workshops with our collaborators in other institutions, including Argonne National Lab.    

We will have a Illinois Tech YouTube channel for yearly RTG workshops, and minicourses on Data Science for Dynamical Systems, Advances in Multiscale Modeling, and Introduction to Quantum Dynamics offered by distinguished RTG visitors. 
  


\subsection{Postdoctoral training}

\textbf{Building personal research program}. A goal for postdoctoral fellows is to   build a well-defined research program, with some teaching experience and mentoring expertise for undergraduate and graduate students.  All postdoctoral fellows will be assisted on their applications to jobs post-RTG.

\textbf{Faculty mentoring}. Each RTG postdoctoral fellow will have two faculty mentors. One mentor will center on research activities
(publications, research presentations, mentoring graduate students). A second faculty member will
mentor on teaching by way of classroom visits and as with graduate students advising on suitable Center for Learning Innovation workshops they should attend. 


\subsection{Outreach}
 
\textbf{The Socially Responsible Modeling, Computation, and Design
(SoReMo) Initiative}

SoReMo advocates for ethical, equitable approaches in computation,
modeling, and design, contributing to the common good of Chicago and
beyond through research and education initiatives at Illinois Tech.

SoReMo holds the forum,  a public bimonthly seminar to help us advance
our mission by shaping conversations and providing guidance to our
student Fellows. Learn from international expert speakers from academia
and industry about cutting edge topics in the area of social responsibility.

SoReMo is also offering to support students with semester-long fellowships.   

\textbf{Workshop on Diversity, Equity and Inclusion (DEI)}
We plan to co-organize this DEI workshop with Mathematics and Computer Science Division at   Argonne National Lab. 

\subsection{Conference participation}
The RTG will provide travel support to RTG students and postdoctoral fellows to attend and present
papers at conferences relevant to their research. We estimate for RTG students to receive support to one
conference per year in the final 2 years of their studies and one conference per year to postdoctoral fellows.
In addition, we will have regular participation of the  RTG team, including undergrads, in the  activities at the NSF Institute for Mathematical and Statistical Innovation (IMSI),  Argonne activities, SIAM Central Section meetings, Midwest Dynamical Systems Conferences, Midwest Numerical Analysis Day, and Midwest Probability Colloquium. This will be in the form of organizing symposia and RTG Seminar
Within the current weekly graduate seminar, we will insert an RTG component every other week. It
will feature talks, mini-courses, and guest lectures aimed at fostering collaboration between all RTG teams.
Consistent with the ongoing format, the responsibility of running it will be shared by RTG graduate students.
Once a year, the RTG-graduate team will organize a seminar under the umbrella of the Illinois Tech 
SIAM Student Chapter. 



\subsection{Recruitment and Retention Plan}

  \subsubsection*{RTG graduate students:} 
To increase the number of PhD students, in addition to the increased TA lines, we also need to improve the pool of applicants in all its aspects - numbers, diversity, and quality. As majority of our applicants tend to be international, we have to work on improving our outreach to colleges and universities in the US. In the past 3 years our applicant pool has numbered around 30-35 each year (for 2-5 supported TA/RA positions). We aim to improve the applicant numbers, the proportion of US applicants, as well as applicants from underrepresented minorities in STEM through faculty and university outreach. With this improvement in total student numbers, we can increase the number of PhD students graduating each year.
Undoubtedly, a major factor in attracting strong PhD students is the stipend. As of Fall 2020, the stipend for PhD students constitutes \$18.5K per AY, which is about 20\% below the average stipend at peer mathematics departments, and significantly below the stipend offered by universities in Chicagoland. We aim to gradually increase the stipend over the next five years to reach the competitive level of \$27K per AY adequate for Chicagoland living expenses. In addition, the department will raise funds for a scholarship (tuition and stipend) awarded annually to an outstanding PhD candidate.

Improve diversity of the student body by reaching out to underrepresented minorities in STEM by, say, sending a faculty representative to recruit at the annual SACNAS meeting event, AMS joint meeting, and AWM-type events.

 
  \subsubsection*{RTG undergraduate students:} 
 To increase this number and to get better prepared and diverse incoming undergraduate students,
we have to establish pathways of engagement between the department and the high schools around Chicago area through a variety of consistent outreach activities as discussed in Section 2.6. Closer coordination with the Illinois Tech undergraduate admissions office is also needed to build awareness of our majors. Illinois Tech also needs to increase its visibility nationally through a marketing campaign which would be beneficial for our recruiting efforts nationally.
To help attract and retain talented and motivated students, we have to provide opportunities for exploring mathematics outside the classroom. At its core, this can be achieved with an active student organization such as the AWM and the SIAM student chapter which are both being run by students with guidance of faculty members. Both these student groups need to continue their activities such as organizing lectures by faculty both in and out of Illinois Tech, alumni panels, social events, and more. Mathematics research can motivate and encourage students to see the influence of mathematics all around us as well train them to take the first steps towards making their own contributions to new and exciting mathematics. Several of our faculty members have done successful research with undergraduates, leading to publications and undergraduate awards from the MAA and the College of Science. Students are also encouraged to participate in competitive REUs (research experiences for undergraduates) outside Illinois Tech, the famous study-abroad program in Budapest, and international competitions such as MCM (Mathematical Contest in Modeling). Several of our students have gone on from these REUs, study programs, and MCM to graduate schools in excellent universities like NYU, UCLA, UIUC, Notre Dame, etc.




%%%%%%%%%%%%%%%%%%%%%%%%%%%%%%%%%%%%%%%%%%%%%%%%%%%%%%%%
\subsection{Performance Assessment Plan  }


 In the assessment plan section, the proposal's goals must be clearly stated. This is necessary so that the National Science Foundation can verify, at the conclusion of the grant or another specified time, that the goals have been reached. This can be done if the goals are numerical, but other types of goals are acceptable as long as verification is not to be based on anecdotal evidence. The proposal's reviewers should have a clear picture of the present status of the research group's activity and how the activity would be enhanced should an award be made. This will be an important part of the review process.

The structure and set of activities of this RTG presents a natural way to assess and monitor performance.
We will specifically collect and evaluate data on the following metrics:

\textbf{Undergraduate participation}\\
• number of students participating in the monthly Undergraduate meetings\\
• number of undergraduate students participating in the monthly meetings that apply and enroll in summer REUs\\
• number of students participating in the Summer SMU-RTG program\\
• number of students who took part in our RTG activity that apply and enroll in a graduate program\\

\textbf{Graduate participation}\\
• number of internship experiences of RTG participants\\
• number of graduate degrees awarded to RTG graduate students and their time to graduation\\
• number of RTG graduate students that applied for postdoctoral fellowships\\
• a survey recording the first appointment post-PhD completion\\

\textbf{Postdoctoral fellows}\\
• a survey recording next employment post-RTG postdoctoral fellowship

\textbf{Demographics}\\
• number of under-represented minorities and women recruited to each part of the RTG program

\textbf{Scholarly output}\\
• number of publications in scholarly journals associated with the RTG program\\
• number of RTG-related presentations at conferences and other institutions\\
• number of proposals and success rates directly connected to RTG teams/research activities

 


%%%%%%%%%%%%%%%%%%%%%%%%%%%%%%%%%%%%%%%%%%%%%%%%%%%%%%%%
\subsection{Organization and Management Plan }
 
The management plan submitted in the proposal must contain a description of the actions that will be taken to achieve the goals set in the assessment plan. One basis for judging proposals will be the goals set and the likelihood that the actions described in the management plan will achieve them. This section should also contain information on the plans to recruit and retain U.S. students and members of underrepresented groups. 
 
\subsubsection*{Management Plan:}
 The PI and 4 Co-PIs will oversee the running of the program. A RTG Director will be appointed. For
this to work effectively, the Director will have a course release. The tentative rotation has Duan as
RTG director on years 1, Hickernell on year 2,    Liu on year 3,    Duan and Maulik for year 4, and Duan and Pelsmajer on year 5.   Together we     will oversee summer activities. The Director will oversee
the budget and assist in the running of RTG activities during the academic year. 


 \subsubsection*{Dissemination:} 
 We will create an RTG website that will include a schedule of activities, publications and highlight
distinctions received by RTG participants. The website will link to other items of interest and a link will
be created for recruitment purposes. Faculty giving invited talks, colloquia or other presentations, will be
asked to include a slide advertising our RTG.

 
 \subsubsection*{Post-RTG plan:}
With the support of the upper administration we will maintain the postdoctoral program with three
fellows each with a 1-1 teaching load. The teaching component for RTG graduate students represents
an increase in the number of classes taught by PhD students. Understanding the importance of having
some teaching experience on their resume, we will work with the upper administration to offer both RTG
stipends and teaching opportunities to all of our post-RTG PhD students. On the undergraduate front, we
will maintain numbers in terms of Camras scholars and BS students applying to graduate schools. Based on
our final assessment of the summer program, if we see merit, we will apply for a follow-up REU proposal.
For an overall assessment of the RTG we will keep contact for the next few years of all participants.



%%%%%%%%%%%%%%%%%%%%%%%%%%%%%%%%%%%%%%%%%%%%%%
\section{Broader Impacts} 
A major goal  of this RTG project is to increase US citizens, nationals and permanent residences in the workforce of  mathematical sciences and related areas. Especially, this project   contributes to  well-trained  data-conscious, diversified    workforce with quantum literacy.  The quantum literacy is urgent and timely for the new  generation of mathematical scientists entering academia, industry and government.

This vertically integrated training program 
  involves women, minority, and members of underrepresented groups. Workshops, minicourses, learning innovation experiences, and research findings will be publicly available on our dedicated RTG website, YouTube, GitHub, and academic journals. Improving written and oral communication skills is required of all trainees.   
 

New courses, research and teaching innovation summaries, and research findings will be publicly available on our RTG website, YouTube, GitHub,
journals, as well as our universities permanent data storage.  
 
 In addition to mathematical expertise, trainees will learn how to better   communicate their results to non-technical audiences.
 
 
 %%%%%%%%%%%%%%%%%%%%%%%%%%%%%%%%%%%%%%%%%%% 
\section{Results from Prior NSF Support}

\noindent\underline{{\bf PI Duan:\,}} Duan and Xiaofan Li received prior support from:
{\it DMS-1620449, 09/15/2016 - 08/31/2020, \$210,000. Theoretical and Numerical Studies of Nonlocal Equations Derived from Stochastic Differential Equations (SDEs) with L\'evy Noises}. Intellectual merit: This project develops efficient numerical techniques for investigating the macroscopic quantities that can help understand the dynamics of SDEs
with $\alpha$-stable L\'evy noises, publications include \cite{Chen, WuFuDuan2019, YangDuan2020,ZhangZhuanDuan,WCDKL,CAI2019166,Cai_2017,IJNAM-17-151}. Broader impacts: One African American female Ph.D. student Julienne Kabre, and three other PhD students were supported  by the project. 
 
 \medskip   
 
\noindent{\bf PI Duan} has the joint NSF grant (with Xiaofan Li)  DMS-1620449: Theoretical and numerical studies of nonlocal equations derived from stochastic differential equations with L\'evy noises, \$210,000, 2016--2020.

\noindent{\textbf{Intellectual Merit of Prior Support}:}


(i)  \textbf{Numerical Methods}:
    We have developed a convergent and fast algorithm to compute macroscopic quantities from stochastic differential equations  with a-stable L\'evy noises. The numerical results on the mean   exit time, escape probability,  and the probability densities have been obtained.

(ii)  \textbf{Theoretical Analysis}:
We have obtained results on  boundary regularity, interior regularity and maximal principles for   nonlocal partial differential equations for the mean   exit time,   escape probability and Fokker-Planck equation for stochastic dynamical systems with L\'evy noise. This has been inspired also by numerical scheme stability requirement of these nonlocal equations. We have further established a  low-dimensional reduction for a slow-fast data assimilation system with non-Gaussian a-stable L\'evy noise (when only slow variables are observable) via stochastic averaging and slow manifolds.

(iii)   \textbf{Applications}:
   We have revealed certain  dynamical features  for  the evolution of concentration in a genetic regulation model and a
   FitzHugh–Nagumo   neuronal model,   under additive and multiplicative     L\'evy fluctuations, by examining and computing mean exit time, escape probability, and maximal likely trajectories (based on nonlocal Fokker-Planck equations for the stochastic systems).


 \noindent{\textbf{Broader Impact of Prior Support}:}

 One African American female Ph.D. student, Julienne Kabre, was supported as full-time RA by the project. Three international students were supported during the summer of 2017 by the project.    Another Ph.D. student Senbao Jiang (from Courant Inst-NYU)  has been supported as research assistant under the grant. He has started to develop the numerical algorithm for solving the two-dimensional integro-differential equation induced by the $\alpha$-stable L\'evy noise.
 
 
\medskip

\noindent{\textbf{Publications Resulting from the Prior NSF Support}:}\\
 \cite{Wang2018NumericalAF, ChenXL2020, GaoTing2016, Gao2016, ZhangZhuanDuan, ZhengDuan2017, DannyTesfay,QiaoDuan2018,YangDuanWiggins2020,ZhengYY2020}.




 




\noindent\textbf{Chun Liu} is the PI on the NSF grants: DMS-1714401,  Topics in Complex Fluids and Biophysiology: the Energetic Variational Approaches, \$349,934, 7/1/2017 - 6/31/2021 and 
{NSF DMS-1950868, Collaborative Research: Multi-Scale Modeling and Numerical Methods for Charge Transport in Ion Channels, \$160,000.00, 8/15/2020 - 7/31/2023.}

\noindent\textbf{Intellectual merit:}  
% Chun Liu is the PI on the NSF grants:
% DMS-1412005, Energetic Variational Approaches in Complex Fluids and Electrophysiology, \$335,000.00,  8/1/2014 - 7/31/2017.
%grant #2024246 from the United States–Israel Binational Science Foundation (BSF), Israel .
PI Liu's research work during the last funding period was focused on the mathematical problems
arising from the modeling of the elastic complex fluids, together
with their physical, biological and engineering applications.
During that period, the PI had more than 40 peer-reviewed papers published in both mathematics and  other disciplinary journals. A general energetic variational framework had been established for various materials
involving multi-physic and multi-scale phenomenon,  in particular, viscoelasticity and polymers;  ionic fluids and general diffusions.
Some of the accomplishments include: (1) Application of the unified energetic variational framework for a wide class of complex fluids, including ion transport, multi-phase flow,
multi-component flow as well as systems involving moving  surfaces \cite{HuLiLi18,
yang_thermodynamically_2018,benesova_existence_2018, deng_largest_2017,xu_strong_2017,benesova_existence_2018,liu_energetic_2019, Kirshtein2020}, the non-isothermal effects \cite{de2019non,liu2018non, hsieh2020global, Jan-Eric}
and chemical reactions \cite{terebus2018discrete, wang2020field}. 
(2) Papers \cite{liu_energetic_2019, epshteyn2019large, Jan-Eric, hsieh2020global} presenting the analysis of the well-posedness, stability and singularity
for those dynamic systems. (3) Construction of corresponding stable and energy preserving numerical schemes \cite{duan_numerical_2019,xu_numerical_2019,duan_numerical_2019-2, wu2019energetic, liu2020lagrangian, liu2020variational, liu2020structure,duan2020structure}.
 (4)  Application of energetic variational approach to dynamics of ion channel and cell motions \cite{horng_continuum_2019,gavish_bistable_2018, liu_generalized_2017},
and related analytical studies \cite{WaLiTa17, hsieh2020global} with incorporation of thermal effects into these approaches \cite{liu2018non, wu2019energetic, hsieh2020global}.  (5) Development of new coarse grain  methods, including moment closure and projections, to study models involving multiple time scales
\cite{ma_fluctuation-dissipation_2017,ma_coarse-graining_2019}.

% [\color{red} I just noticed that some publication before 2017]}

\noindent{\bf Broader impacts:}  PI Liu had worked with
3 postdocs and 3 graduate students, along with several undergraduate students 
on related projects of this proposal. 
% The group also involves  s.
The students learned mathematical, physical and biological
aspects of the theory, while actively collaborating  with researchers in those fields.
They also attended and presented their work in workshops and conferences.


%%%%%%%%%%%%%%%%%%%%%%%%%%%%%%%%%%%%%%%%%%%%%%%%%%%%%%%%%%%%%%%%%%%%%%%%%%%%%%%%%%%

\noindent PI \textbf{Hickernell} led NSF-DMS-1522687 Stable, Efficient, Adaptive Algorithms for Approximation and Integration,
		\$270,000, 8/1/2015 -- 7/31/2018 along with Gregory E.\ Fasshauer (co-PI) and  Sou-Cheng Terrya Choi (female, Senior Personnel).  Other major contributors were Hickernell's research students: 
		six PhDs earned (two female), one current PhD student, and three MSs earned (two female).
Articles, theses,
software, and preprints supported in
part by this
grant
include
\cite{ala_augmented_2017,
	ChoEtal17a,
	ChoEtal20a,
	Din15a,
	DinHic20a,
	GilEtal16a,
	Hic17a,
	HicJag18b,
	HicJim16a,
	HicEtal18a,
	HicEtal17a,
	HicKriWoz19a,
	RatHic19a,
	GilJim16b,
	JimHic16a,
	JohFasHic18a,
	Li16a,
	Liu17a,
	MarEtal18a,
	mccourt_stable_2017,
	MCCEtal19a,
	mishra_hybrid_2018,
	MisEtal19a,
	rashidinia_stable_2016,
	rashidinia_stable_2018,
	Zha18a,
	Zha17a,
	Zho15a,
	ZhoHic15a}.

%%%%%%%%%%%%%%%%%%%%%%%%%%%%%%%%%%%%%%%%%%%%%%%%%%%%%%%%%%%%%%%%%%%%%%%%%%%%%%%%%%%
\noindent \textbf{Intellectual Merit:}
Hickernell and collaborators developed adaptive algorithms for univariate integration, function approximation, and optimization \cite{ChoEtal17a,HicEtal14b, Din15a, Ton14a, Zha18a}.
Hickernell and collaborators developed globally adaptive algorithms for approximating multivariate integrals over the unit cube based on low discrepancy sequences \cite{HicJim16a,HicEtal17a,JimHic16a,RatHic19a}.  The error bounds underlying these adaptive cubatures rely on the Fourier coefficients of the sampled function values, either by tracking their decay rate or by using them to construct Bayesian credible intervals. The cost to bound the error using function data is essentially the same as order the cost to approximate the integral. 
Hickernell and collaborators investigated function approximation problems for Banach spaces defined by series representations \cite{DinHic20a,DinEtal20a}.  For all of these adaptive algorithms, defining a suitable cone of functions to be integrated, approximated, or optimized is key

\noindent \textbf{Broader Impacts:}
Publications arising from this project are reference above.  
The project personnel spoke at many academic conferences and gave colloquium/seminar talks to mathematics and statistics departments and at several conferences.  Hickernell co-organized the 2016 Spring Research Conference and was a program leader for the SAMSI 2017--18 Quasi-Monte Carlo (QMC) Program.   He received the 2016 Joseph F.\ Traub Prize for Achievement in Information-Based Complexity. The algorithms from this research have been implemented in our open source MATLAB Guaranteed Automatic Integration Library (GAIL), which is used in the yearly graduate course in Monte Carlo methods.  Hickernell and the other senior personnel have mentored a number of research students associated with this project, including several female students.  More than a dozen undergraduates have been mentored and several have proceeded to graduate study in the mathematical sciences. 


\noindent PI \textbf{Pelsmajer}: No current NSF funding or   awards with an end data within the past five years. 




\newpage
\pagenumbering{arabic}
\renewcommand{\thepage} {\arabic{page}}

\bibliographystyle{abbrv}
% \bibliographystyle{plain}
%\bibliographystyle{natbib}
\bibliography{asi,ref_art,stochas,ml_ref,datasci, VD, quantum, FJHown23, FJH23}

\end{document}    % End %


%%%%%%%%%%%%%%%%%%%%%%%%%%%%%%%
%%%%%%%%%%%%%%%%%%%%%%%%%%%%%
%%%%%%%%%%%%%%%
%%%%%%%%%%%%%%%%%%%%%%%%%%%%%%%%%%%%%%%%%%
\section{Supplementary Documentation}
 

\subsection{a. Letters of Collaboration}

--- from Argonne Lab


Signed letters of collaboration by the institution and other sources in support of the project should be included. If industrial or government laboratory internships are planned, letters indicating the willingness of the external organization and of individual external mentors (if known) to participate should also be included. These documents should be scanned and uploaded into the supplementary documentation section.
The letters of collaboration are meant to explain how the institution and the collaborating sites will provide an environment that supports the proposed research and training activities. It is acceptable for a letter of collaboration to briefly mention specific activities supported by the collaboration and listed in the proposal; however, each letter is limited to one page. Letters of recommendation or endorsement are not allowed.

\subsection{b. Trainee Data} 

All applicants are strongly encouraged to supply the following data. Note that data is requested for the group submitting the proposal, not for the entire department. For new RTG proposals, data should be included for the past five years. For a renewal of an existing RTG grant, data should be included for the past ten years.
A list of Ph.D. recipients, along with each individual's baccalaureate institution, time-to-degree, post-Ph.D. placement, and thesis advisor.
A list of postdoctoral associates (including holders of named instructorships and 2- or 3-year terminal assistant professors), their Ph.D. institutions, postdoctoral mentors, and post-appointment placements.
The dollar amount of funding by federal agencies for Research Experiences for Undergraduates (REUs), graduate students, and postdoctoral associates.





\subsubsection{A list of graduate students}

Romit:

Summer graduate students through Argonne:
Suraj Pawar, Oklahoma State University, Research Aide, 2020
Dominic Skinner, MIT, NSF MSGI, 2020
Alec Linot, University of Wisconsin Madison, Wallace Givens Associate, 2021
Sahil Bhola, University of Michigan, Research aide, 2021

\subsubsection{A list of postdoctoral associates}
  
  (including holders of named
     instructorships and 2- or 3-year terminal assistant professors),
     their Ph.D. institutions, postdoctoral mentors, and post-appointment
     placements.

     I know those from 2017. Can you check and keep it as a record?

     1) From Duan's group
     
     Jessica Xia 
     
     
     2) From Chun’s group:

     Stefan Metzger: 8/2017 — 8/2018, FAU ( Friedrich-Alexander
     University, Erlangen-Nurnberg), FAU
     Qing Cheng: 1/2019 — 1/2021, Xiamen University, Purdue University
     Yiwei Wang: 8/2018 — present, Peking University, IIT

     2) From Igor’s group

     Hyun-Jung Kim, 8/2018  — 8/2020, USC, UCSB

     3) From Sonja’s group

     Sara Jamshidi: 8/2018 — 8/2020, Penn State, Lake Forest College

     4) From Shuwang’s group

     Pedro Anjos: 8/2019 — present, Federal University of Pernambuco, IIT.




\subsection{Quantitative Demographic Data}
 


%%%%%%%%%%%%%%%%%%%%%%%%%%%%%%%%%%%%%%%%%%%%%%%%%%%%%%%% 
